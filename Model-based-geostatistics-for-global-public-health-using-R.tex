% Options for packages loaded elsewhere
\PassOptionsToPackage{unicode}{hyperref}
\PassOptionsToPackage{hyphens}{url}
\PassOptionsToPackage{dvipsnames,svgnames,x11names}{xcolor}
%
\documentclass[
  letterpaper,
]{krantz}

\usepackage{amsmath,amssymb}
\usepackage{iftex}
\ifPDFTeX
  \usepackage[T1]{fontenc}
  \usepackage[utf8]{inputenc}
  \usepackage{textcomp} % provide euro and other symbols
\else % if luatex or xetex
  \usepackage{unicode-math}
  \defaultfontfeatures{Scale=MatchLowercase}
  \defaultfontfeatures[\rmfamily]{Ligatures=TeX,Scale=1}
\fi
\usepackage{lmodern}
\ifPDFTeX\else  
    % xetex/luatex font selection
\fi
% Use upquote if available, for straight quotes in verbatim environments
\IfFileExists{upquote.sty}{\usepackage{upquote}}{}
\IfFileExists{microtype.sty}{% use microtype if available
  \usepackage[]{microtype}
  \UseMicrotypeSet[protrusion]{basicmath} % disable protrusion for tt fonts
}{}
\makeatletter
\@ifundefined{KOMAClassName}{% if non-KOMA class
  \IfFileExists{parskip.sty}{%
    \usepackage{parskip}
  }{% else
    \setlength{\parindent}{0pt}
    \setlength{\parskip}{6pt plus 2pt minus 1pt}}
}{% if KOMA class
  \KOMAoptions{parskip=half}}
\makeatother
\usepackage{xcolor}
\setlength{\emergencystretch}{3em} % prevent overfull lines
\setcounter{secnumdepth}{5}
% Make \paragraph and \subparagraph free-standing
\ifx\paragraph\undefined\else
  \let\oldparagraph\paragraph
  \renewcommand{\paragraph}[1]{\oldparagraph{#1}\mbox{}}
\fi
\ifx\subparagraph\undefined\else
  \let\oldsubparagraph\subparagraph
  \renewcommand{\subparagraph}[1]{\oldsubparagraph{#1}\mbox{}}
\fi

\usepackage{color}
\usepackage{fancyvrb}
\newcommand{\VerbBar}{|}
\newcommand{\VERB}{\Verb[commandchars=\\\{\}]}
\DefineVerbatimEnvironment{Highlighting}{Verbatim}{commandchars=\\\{\}}
% Add ',fontsize=\small' for more characters per line
\usepackage{framed}
\definecolor{shadecolor}{RGB}{241,243,245}
\newenvironment{Shaded}{\begin{snugshade}}{\end{snugshade}}
\newcommand{\AlertTok}[1]{\textcolor[rgb]{0.68,0.00,0.00}{#1}}
\newcommand{\AnnotationTok}[1]{\textcolor[rgb]{0.37,0.37,0.37}{#1}}
\newcommand{\AttributeTok}[1]{\textcolor[rgb]{0.40,0.45,0.13}{#1}}
\newcommand{\BaseNTok}[1]{\textcolor[rgb]{0.68,0.00,0.00}{#1}}
\newcommand{\BuiltInTok}[1]{\textcolor[rgb]{0.00,0.23,0.31}{#1}}
\newcommand{\CharTok}[1]{\textcolor[rgb]{0.13,0.47,0.30}{#1}}
\newcommand{\CommentTok}[1]{\textcolor[rgb]{0.37,0.37,0.37}{#1}}
\newcommand{\CommentVarTok}[1]{\textcolor[rgb]{0.37,0.37,0.37}{\textit{#1}}}
\newcommand{\ConstantTok}[1]{\textcolor[rgb]{0.56,0.35,0.01}{#1}}
\newcommand{\ControlFlowTok}[1]{\textcolor[rgb]{0.00,0.23,0.31}{#1}}
\newcommand{\DataTypeTok}[1]{\textcolor[rgb]{0.68,0.00,0.00}{#1}}
\newcommand{\DecValTok}[1]{\textcolor[rgb]{0.68,0.00,0.00}{#1}}
\newcommand{\DocumentationTok}[1]{\textcolor[rgb]{0.37,0.37,0.37}{\textit{#1}}}
\newcommand{\ErrorTok}[1]{\textcolor[rgb]{0.68,0.00,0.00}{#1}}
\newcommand{\ExtensionTok}[1]{\textcolor[rgb]{0.00,0.23,0.31}{#1}}
\newcommand{\FloatTok}[1]{\textcolor[rgb]{0.68,0.00,0.00}{#1}}
\newcommand{\FunctionTok}[1]{\textcolor[rgb]{0.28,0.35,0.67}{#1}}
\newcommand{\ImportTok}[1]{\textcolor[rgb]{0.00,0.46,0.62}{#1}}
\newcommand{\InformationTok}[1]{\textcolor[rgb]{0.37,0.37,0.37}{#1}}
\newcommand{\KeywordTok}[1]{\textcolor[rgb]{0.00,0.23,0.31}{#1}}
\newcommand{\NormalTok}[1]{\textcolor[rgb]{0.00,0.23,0.31}{#1}}
\newcommand{\OperatorTok}[1]{\textcolor[rgb]{0.37,0.37,0.37}{#1}}
\newcommand{\OtherTok}[1]{\textcolor[rgb]{0.00,0.23,0.31}{#1}}
\newcommand{\PreprocessorTok}[1]{\textcolor[rgb]{0.68,0.00,0.00}{#1}}
\newcommand{\RegionMarkerTok}[1]{\textcolor[rgb]{0.00,0.23,0.31}{#1}}
\newcommand{\SpecialCharTok}[1]{\textcolor[rgb]{0.37,0.37,0.37}{#1}}
\newcommand{\SpecialStringTok}[1]{\textcolor[rgb]{0.13,0.47,0.30}{#1}}
\newcommand{\StringTok}[1]{\textcolor[rgb]{0.13,0.47,0.30}{#1}}
\newcommand{\VariableTok}[1]{\textcolor[rgb]{0.07,0.07,0.07}{#1}}
\newcommand{\VerbatimStringTok}[1]{\textcolor[rgb]{0.13,0.47,0.30}{#1}}
\newcommand{\WarningTok}[1]{\textcolor[rgb]{0.37,0.37,0.37}{\textit{#1}}}

\providecommand{\tightlist}{%
  \setlength{\itemsep}{0pt}\setlength{\parskip}{0pt}}\usepackage{longtable,booktabs,array}
\usepackage{calc} % for calculating minipage widths
% Correct order of tables after \paragraph or \subparagraph
\usepackage{etoolbox}
\makeatletter
\patchcmd\longtable{\par}{\if@noskipsec\mbox{}\fi\par}{}{}
\makeatother
% Allow footnotes in longtable head/foot
\IfFileExists{footnotehyper.sty}{\usepackage{footnotehyper}}{\usepackage{footnote}}
\makesavenoteenv{longtable}
\usepackage{graphicx}
\makeatletter
\def\maxwidth{\ifdim\Gin@nat@width>\linewidth\linewidth\else\Gin@nat@width\fi}
\def\maxheight{\ifdim\Gin@nat@height>\textheight\textheight\else\Gin@nat@height\fi}
\makeatother
% Scale images if necessary, so that they will not overflow the page
% margins by default, and it is still possible to overwrite the defaults
% using explicit options in \includegraphics[width, height, ...]{}
\setkeys{Gin}{width=\maxwidth,height=\maxheight,keepaspectratio}
% Set default figure placement to htbp
\makeatletter
\def\fps@figure{htbp}
\makeatother
\newlength{\cslhangindent}
\setlength{\cslhangindent}{1.5em}
\newlength{\csllabelwidth}
\setlength{\csllabelwidth}{3em}
\newlength{\cslentryspacingunit} % times entry-spacing
\setlength{\cslentryspacingunit}{\parskip}
\newenvironment{CSLReferences}[2] % #1 hanging-ident, #2 entry spacing
 {% don't indent paragraphs
  \setlength{\parindent}{0pt}
  % turn on hanging indent if param 1 is 1
  \ifodd #1
  \let\oldpar\par
  \def\par{\hangindent=\cslhangindent\oldpar}
  \fi
  % set entry spacing
  \setlength{\parskip}{#2\cslentryspacingunit}
 }%
 {}
\usepackage{calc}
\newcommand{\CSLBlock}[1]{#1\hfill\break}
\newcommand{\CSLLeftMargin}[1]{\parbox[t]{\csllabelwidth}{#1}}
\newcommand{\CSLRightInline}[1]{\parbox[t]{\linewidth - \csllabelwidth}{#1}\break}
\newcommand{\CSLIndent}[1]{\hspace{\cslhangindent}#1}

\renewcommand{\and}{\\}
\newtheorem{theorem}{Theorem}[chapter]
\newtheorem{exercise}{Exercise}[chapter]
\newtheorem{example}{Example}[chapter]
\newtheorem{definition}{Definition}[chapter]
%%\newtheorem{proof}{Proof}
\newenvironment{proof}[1][Proof]%
%  \topsep6\p@\@plus6\p@ \trivlist
{\par\addvspace{6pt}\normalfont {\bfseries #1}\hskip\labelsep\ignorespaces\itshape}
{\par\addvspace{6pt}}
\makeatletter
\makeatother
\makeatletter
\@ifpackageloaded{bookmark}{}{\usepackage{bookmark}}
\makeatother
\makeatletter
\@ifpackageloaded{caption}{}{\usepackage{caption}}
\AtBeginDocument{%
\ifdefined\contentsname
  \renewcommand*\contentsname{Table of contents}
\else
  \newcommand\contentsname{Table of contents}
\fi
\ifdefined\listfigurename
  \renewcommand*\listfigurename{List of Figures}
\else
  \newcommand\listfigurename{List of Figures}
\fi
\ifdefined\listtablename
  \renewcommand*\listtablename{List of Tables}
\else
  \newcommand\listtablename{List of Tables}
\fi
\ifdefined\figurename
  \renewcommand*\figurename{Figure}
\else
  \newcommand\figurename{Figure}
\fi
\ifdefined\tablename
  \renewcommand*\tablename{Table}
\else
  \newcommand\tablename{Table}
\fi
}
\@ifpackageloaded{float}{}{\usepackage{float}}
\floatstyle{ruled}
\@ifundefined{c@chapter}{\newfloat{codelisting}{h}{lop}}{\newfloat{codelisting}{h}{lop}[chapter]}
\floatname{codelisting}{Listing}
\newcommand*\listoflistings{\listof{codelisting}{List of Listings}}
\makeatother
\makeatletter
\@ifpackageloaded{caption}{}{\usepackage{caption}}
\@ifpackageloaded{subcaption}{}{\usepackage{subcaption}}
\makeatother
\makeatletter
\@ifpackageloaded{tcolorbox}{}{\usepackage[skins,breakable]{tcolorbox}}
\makeatother
\makeatletter
\@ifundefined{shadecolor}{\definecolor{shadecolor}{rgb}{.97, .97, .97}}
\makeatother
\makeatletter
\makeatother
\makeatletter
\makeatother
\ifLuaTeX
  \usepackage{selnolig}  % disable illegal ligatures
\fi
\IfFileExists{bookmark.sty}{\usepackage{bookmark}}{\usepackage{hyperref}}
\IfFileExists{xurl.sty}{\usepackage{xurl}}{} % add URL line breaks if available
\urlstyle{same} % disable monospaced font for URLs
\hypersetup{
  pdftitle={Model-based geostatistics for global public health using R},
  pdfauthor={Emanuele Giorgi; Claudio Fronterre},
  colorlinks=true,
  linkcolor={blue},
  filecolor={Maroon},
  citecolor={Blue},
  urlcolor={Blue},
  pdfcreator={LaTeX via pandoc}}

\title{Model-based geostatistics for global public health using R}
\author{Emanuele Giorgi \and Claudio Fronterre}
\date{2023-03-03}

\begin{document}
\maketitle
\ifdefined\Shaded\renewenvironment{Shaded}{\begin{tcolorbox}[breakable, sharp corners, boxrule=0pt, enhanced, frame hidden, interior hidden, borderline west={3pt}{0pt}{shadecolor}]}{\end{tcolorbox}}\fi

\renewcommand*\contentsname{Table of contents}
{
\hypersetup{linkcolor=}
\setcounter{tocdepth}{2}
\tableofcontents
}
\bookmarksetup{startatroot}

\hypertarget{preface}{%
\chapter*{Preface}\label{preface}}
\addcontentsline{toc}{chapter}{Preface}

\markboth{Preface}{Preface}

Its companion book ``Model-based geostatistical for global public
health'\,' by Peter J. Diggle (2019) is a strongly recommended
complementary read, as you work your way through this book.

\bookmarksetup{startatroot}

\hypertarget{introduction}{%
\chapter{Introduction}\label{introduction}}

The book provides shows how to carry out model-based geostatistical
analysis of public health data using the \texttt{RiskMap} R package. In
this introductory chapter, we explain what are the pre-requisites for
using this book and its learning objectives. We also explain what
software should be installed and how. Finally, we give a brief overview
of the class of models covered in this book, and the examples that will
be used to illustrate the methods and use of software.

\hypertarget{objectives-of-this-book}{%
\section{Objectives of this book}\label{objectives-of-this-book}}

The overall aim of this book is to provide you with the skills to
perform a geostatistical analysis of a data-set using the R software
environment. As you work your way through the book, you will learn to:

\begin{itemize}
\tightlist
\item
  explore geostatistical data-sets using graphical procedures and
  summary statistics;
\item
  formulate and fit geostatistical models using the maximum likelihood
  estimation method;
\item
  carry out prediction of health outcomes at different spatial scales;
\item
  visualize and interpret the results from geostatistical models;
\item
  model the relationships between spatially referenced risk factors and
  the health outcome of interest;
\item
  validate the assumptions of geostatistical models and assess their
  predictive performance.
\end{itemize}

Although the focus of this book is on public health, the statistical
ideas, as well as the software used, can also be applied for the
analysis of geostatistical data-sets arising from other scientific
fields.

\hypertarget{pre-requisites-for-using-this-book}{%
\section{Pre-requisites for using this
book}\label{pre-requisites-for-using-this-book}}

To effectively understand and use the material presented in this book,
it is expected that you should possess prior knowledge of basic
probability theory, foundational topics in statistical modelling and R
programming. Below we provide a more detailed explanation of the
pre-requisites for each of these three fields.

\hypertarget{topics-in-probability}{%
\subsection{Topics in probability}\label{topics-in-probability}}

Basics probability theory is important to fully understand the content
of this book. In particular, you should have knowledge of: the general
definition and properties of continuous and discrete distribution; how
the describe the properties of probability distributions through their
mean, variance and skeweness; the concepts of stochastic dependence and
correaltion; the distinction between marginal and conditional
distributions; the basic properties of the Gaussian, Binomial and
Poisson distributions; the definition and properties of the multivariate
Gaussian distribution. The redear can find an extensive explanation and
illustrations with examples of all these topics in Ross (2013).

\hypertarget{topics-in-statistics}{%
\subsection{Topics in statistics}\label{topics-in-statistics}}

Likelihood-based inference (whether frequensist or Bayesian) provides
the theoretical bedrock for the estimation of almost any statistical
model. In this book will focus on maximum likelihood estimation methods
of inference. Extensive use of the notions of point and interval
estimates obtained using the maximum likelihood estimation methods will
be made through the book. Recommended readings include chapters 1, 2 and
4 of Pawitan (2001).

Good prior knowledge of Generalized linear models (GLMs) is essential,
as the geostatistical modelling framework builds on these as an
extension. Before embarking on the use of this book, we thus encourage
you to review the basic theory of GLMs and, in particular, how these are
applied and interpreted. In this book, we will cover examples that will
model continuously measured outcomes and counts. Hence, good
understanding of linear regression modelling and modelling of counts
data using Binomial and Poisson regression should be the main focus of
the review. For comprehensive overview of GLMs and their implementation
in R, we refer you to Dobson and Barnett (2008).

\hypertarget{topics-in-r-programming}{%
\subsection{Topics in R programming}\label{topics-in-r-programming}}

Although this book does not require to possess advanced skills in R
programming, it is important you have good knowledge in the following
topics: creation and manipulation of vectors and matrices; logical
vectors; character vectors; handling of lists and data frame objects;
reading data into R; graphical procedures. A very large amount of freely
available material covering these topics can be found online. Our
recommendation is to start from the manual ``An introduction to R'' of
the Comprehensive R Archive Network available at this link, available at
\href{https://cran.r-project.org/manuals.html}{R manual}.

\hypertarget{obtaining-and-running-the-r-packages}{%
\section{Obtaining and running the R
packages}\label{obtaining-and-running-the-r-packages}}

It is advised that you obtain the latest 64-bit version of R in order to
run the R code of this book. To install R, go to the R website, where
you can download the installer packages for Windows and Mac, and find
instructions for Linux, using binary files.

\begin{itemize}
\tightlist
\item
  \href{https://cran.r-project.org/bin/windows/base/}{Windows}
\item
  \href{https://cran.r-project.org/bin/macosx}{Mac}
\item
  \href{https://cran.r-project.org/bin/linux}{Linux}
\end{itemize}

The list of the R packages used in this book is provided in
Table~\ref{tbl-packages}.

\hypertarget{tbl-packages}{}
\begin{longtable}[]{@{}
  >{\raggedright\arraybackslash}p{(\columnwidth - 2\tabcolsep) * \real{0.3611}}
  >{\raggedright\arraybackslash}p{(\columnwidth - 2\tabcolsep) * \real{0.6389}}@{}}
\caption{\label{tbl-packages}List of the R packages that will be used in
the book with a description of their use in the data analysis. The
packages marked by (E) are essential for the geostatistical analysis.
Those instead marked by (R) are recommended and can be helpful to
overcome issues as described under the column ``Used
for''.}\tabularnewline
\toprule\noalign{}
\begin{minipage}[b]{\linewidth}\raggedright
R packages
\end{minipage} & \begin{minipage}[b]{\linewidth}\raggedright
Used for
\end{minipage} \\
\midrule\noalign{}
\endfirsthead
\toprule\noalign{}
\begin{minipage}[b]{\linewidth}\raggedright
R packages
\end{minipage} & \begin{minipage}[b]{\linewidth}\raggedright
Used for
\end{minipage} \\
\midrule\noalign{}
\endhead
\bottomrule\noalign{}
\endlastfoot
\texttt{RiskMap} (E) & Estimating of geostatistical models and spatial
prediction \\
\texttt{sf} (E) & Handling of spatial data in R \\
\texttt{terra} (E) & Handling of raster files in R \\
\texttt{ggplot2} (E) & Creating maps and exploratory plots \\
\texttt{crsuggest} (R) & Guessing a coordinate reference systems when
unknown \\
\end{longtable}

To install packages in R for the first time, you can use the command
\texttt{install.packages} in the R console, as shown below for the
\texttt{RiskMap} package.

\begin{Shaded}
\begin{Highlighting}[]
\FunctionTok{install.packages}\NormalTok{(}\StringTok{"RiskMap"}\NormalTok{)}
\end{Highlighting}
\end{Shaded}

\hypertarget{sec-examples-ch1}{%
\section{Example data-sets used in the book}\label{sec-examples-ch1}}

The geostatistical data-sets described in this section will be used
throughout the book to illustrate the use of the R packages mentioned in
the previous sections.

Each of the examples data-sets can be loaded from the \texttt{RiskMap}
package, using the command

\begin{Shaded}
\begin{Highlighting}[]
\FunctionTok{data}\NormalTok{(galicia)}
\end{Highlighting}
\end{Shaded}

for the lead concentration data from Galicia,

\begin{Shaded}
\begin{Highlighting}[]
\FunctionTok{data}\NormalTok{(liberia)}
\end{Highlighting}
\end{Shaded}

for the river-blindness data-set,

\begin{Shaded}
\begin{Highlighting}[]
\FunctionTok{data}\NormalTok{(malkenya)}
\end{Highlighting}
\end{Shaded}

for the malaria data in the Western Highlands of Kenya, and

\begin{Shaded}
\begin{Highlighting}[]
\FunctionTok{data}\NormalTok{(anopheles)}
\end{Highlighting}
\end{Shaded}

for the Anopheles mosquitoes data-set.

In the final chapter of this book, we will consider the analysis of
additional data-sets to review the main statistical concepts presented
in this book.

\hypertarget{lead-concentration-in-galicia}{%
\subsection{Lead concentration in
Galicia}\label{lead-concentration-in-galicia}}

\begin{figure}

{\centering \includegraphics[width=3.31in,height=\textheight]{./figures/galicia_ch1.png}

}

\caption{\label{fig-galicia-ch1}Data on the meausred lead concentration
(in micrograms per gram dry weight) in moss samples collected in
Galicia, North-West of Spain.}

\end{figure}

Lead is a heavy metal which, in high concentrations, can cause chronic
damage to living organisms over a long period of time. For this reason
its spread and source must be regularly monitored. To assess the extent
of the contamination in an area, measurements of lead are often taken
from plants. The data here considered (Figure~\ref{fig-galicia-ch1})
consist of 132 locations of moss samples collected in 2000, in and
around Galicia, a region in the North-Western part of Spain. One of the
objectives of this survey was to establish the spatial pattern of lead
concentration in Galicia so as to better identify possible sources of
contamination; fore more information, see Fernández, Rey, and
Carballeira (2000).

In this case, geostatistical modelling can be used to predict the lead
concentration across Galicia and allows to disentangle variation which
is purely random, possibly due to measurement error, and genuine spatial
variation, which is our main object of interest.

This data-set will be used in this book to show how to carry out the
spatial analysis of continuously measured variables using linear
geostatistical models.

\hypertarget{sec-rb-ch1}{%
\subsection{River-blindness in Liberia}\label{sec-rb-ch1}}

\begin{figure}

{\centering \includegraphics[width=3.31in,height=\textheight]{./figures/liberia_ch1.png}

}

\caption{\label{fig-liberia-ch1}River-blindness data from a
cross-sectional survey carried out in Liberia.}

\end{figure}

In low-resource settings, where disease registries are typically absent,
cross-sectional surveys are an essential monitoring tool that enables
the estimation of the disease burden in a population of interest. The
data considered in this example (Figure~\ref{fig-liberia-ch1}) have been
collected as part of an Africa-wide initiative called the Rapid
Epidemiological Mapping of Onchocerchiasis (REMO) carried out in 2011 in
20 African countries (Zouré et al. 2014). The goal of REMO is to
identify areas where river-blindness (or onchocerchiasis), a disease
transmitted by black flies who breed along fast flowing rivers, is still
a public health problem. In this context, it is especially of interest
to identify communities with a prevalence above 20\% and for treatment
is urgently needed.

In this book, we will use data collected from Liberia to model nodule
prevalence, which is based on a alternative and cheaper diagnostic
technique for river-blindness. In the analysis of this data-set, we will
illustrate how to formulate and fit Binomial geostatistical models, and
how these can be used to predict prevalence within a region of interest.

\hypertarget{sec-malaria-ch1}{%
\subsection{Malaria in the Western Kenyan
Highlands}\label{sec-malaria-ch1}}

\begin{figure}

{\centering \includegraphics[width=3.31in,height=\textheight]{./figures/malkenya_ch1.png}

}

\caption{\label{fig-malkenya-ch1}Malaria prevalence data from a
cross-sectional survey carried out in Nyanza Province, in the Western
Highlands of Kenya.}

\end{figure}

Malaria is one of deadliest diseases that affects populations living in
tropical and subtropical countries. It is caused by a parasite of the
genus Plasmodium which is transmitted through the infectious bite of
female Anopheles mosquitoes. In the following chapters, we shall analyse
a data-set from a cross-sectional community survey carried out in July
2010 in Nyanza Province, in the Western Highlands of Kenya (Stevenson
2013).

What distinguishes this from the other examples data-sets is that the
data contain both individual-level and household-level information. The
outcome of interest is the result from a rapid diagnostic test for
malaria which. In the book, we will illustrate how to account for the
the hierarchical structure of the data and the binary nature of the
outcome at each of the stages of the geostatistical analysis.

\hypertarget{sec-mosq-data-ch1}{%
\subsection{\texorpdfstring{\emph{Anopheles gambiae} mosquitoes in
Southern
Cameroon}{Anopheles gambiae mosquitoes in Southern Cameroon}}\label{sec-mosq-data-ch1}}

\begin{figure}

{\centering \includegraphics[width=3.31in,height=\textheight]{./figures/anopheles_ch1.png}

}

\caption{\label{fig-anopheles-ch1}Map of the collected number of
\emph{Anopheles gambiae} mosquitoes in an area of Southern Cameroon.}

\end{figure}

In studies of vector-borne and zoonotic diseases, understanding of the
vector distribution can help to better guide the decision-making process
for the implementation, monitoring and evaluation of control programmes.
\emph{Anopheles gambiae} mosquitoes are one of the main vectors for
malaria transmission in sub-Saharan Africa. Their distribution over
space is affected by several environmental and climatic factors,
including temperature, humidity and vegetation.

The data-set on mosquitoes (Figure~\ref{fig-anopheles-ch1}) that will
use in the book consists of a sub-set taken from a large database (Tene
Fossog et al. 2015). This was assembled in order to understand how the
environment affects the distribution of different species of
\emph{Anopheles} mosquitoes in sub-Saharan Africa. This example data-set
will be used to illustrate the application of Poisson geostatistical
models for mapping mosquitoes abundance.

\hypertarget{sec-geostat-models}{%
\section{Geostatistical problems and geostatistical
models}\label{sec-geostat-models}}

What the examples of the previous section have in common is that, in
each case, the goal of statistical analysis is to draw inferences on an
unobserved spatially continuous surface using data collected from a
finite set of locations. The lead concentration in Galicia, the
prevalence for river-blindness in Liberia and the abundance of \emph{A.
gambiae} mosquitoes in Cameroon can all be represented as spatially
continuous processes that originate from the combined effects of
environmental factors. We denote this class of inferential problems as
\emph{geostatistical problems} for which a solution can be found through
the development and application of suitable \emph{geostatistical
models}, which are the subject of this book.

As one can soon realize, geostatistical problems are not unique to
global health but arise in many other fields of science, including
economics, physics, biology, geology and others. It thus comes to no
surprise that geostatistics was initially developed in the South African
mining industry in the 1950s (Krige 1951). This was then further
developed as a self-contained discipline by Georges Matheron and other
researchers at Fontainebleau, in France (Matheron 1963; Chilès and
Delfiner 2016). In Watson (1971) and Watson (1972) a first connection is
drawn between geostatistics and the prediction of stochastic processes.
However, it is only with Ripley (1981) and then Cressie (1991) that
geostatistics is explicitly brought into a classical statistical
framework for the analysis of spatially referenced data. P. J. Diggle,
Tawn, and Moyeed (1998) coined the term \emph{model-based geostastics}
and introduced this as belonging to the general class of generalized
linear mixed models (Breslow and Clayton 1993), while emphasizing the
use of likelihood-based methods of inference. As in P. J. Diggle, Tawn,
and Moyeed (1998), also in this book, we advocate the application of
model-based geostistical models as a class of parametric statistical
models on which inference can be carried out using either maximum
likelihood estimation or Bayesian methods.

More precisely, our attention will be directed at the class of
\emph{generalized linear geostatistical models}, or GLGM. To formally
specify this, we first define the random variables \(S\), a spatial
stochastic process, and the random variable \(Y= (Y_1, \ldots, Y_n)\)
which correspond to the outcome observed at a set of locations
\(X = (x_1, \ldots, x_n)\). Let us use \([A]\) to denote ``the
distribution of the random variable \(A\)''. To formulate a GLGM, we
should then specify the joint distribution of \(S\) and \(Y\), which we
write as

\begin{equation}\protect\hypertarget{eq-glgm-joint-ch1}{}{
[Y, S] = [S] [Y | S].
}\label{eq-glgm-joint-ch1}\end{equation}

On the right-hand side of the equation above, we have factorized the
joint distribution of \(Y\) and \(S\), as the product between the
marginal distribution of \(S\) and the conditional distribution of \(Y\)
given \(S\). Hence, the formulation of a GLGM can be break down into the
tasks of formulating \([S]\) and \([Y | S]\).

In defining \([S]\), throughout the book, we shall assume that this is a
zero-mean stationary and isotropic Gaussian process. In other words,
these assumptions impose that the joint distribution of
\(S(X) = (S(x_1),\ldots,S(x_n))\), i.e.~the process \(S\) at the sampled
locations \(x_1, \ldots, x_n\), is invariant with respect to rations and
translations of the locations \(X\). In practical terms, the main
implication of this is that, for any pair of locations \(x_i\) and
\(x_j\) the correlation function \(\rho(\cdot)\) between \(S(x_i)\) and
\(S(x_j)\) is purely a function of the Euclidean distance, \(u_{ij}\),
between \(x_i\) and \(x_j\), i.e.

\begin{equation}\protect\hypertarget{eq-correlation-ch1}{}{
{\rm cov}\{S(x_i), S(x_j)\} = \sigma^2\rho(u_{ij}),
}\label{eq-correlation-ch1}\end{equation}

where \(\sigma^2\) is the variance of \(S(x)\) for all \(x\). In Chapter
3, we will look more closely at what type of correlation functions can
be used for \(\rho(\cdot)\) and how these affect our predictive
inferences. Furthermore, the fact that assume the process \(S\) to have
mean zero is because this is process acts as a residual term in our
modelling of \(Y\). This aspect will be reiterated several times in the
following chapters, as it as important implications for the
interpretation of the other components of a geostatistical model, as
well understanding the results of the analysis.

Finally, we model \([Y | S]\), i.e.~the distribution of \(Y\) given
\(S\), is modeled as a set of mutually independent distributions which
belong the exponential family, as defined in classical generalized
linear modelling framework (Nelder and Wedderburn 1972). It then follows
that, we can write \([Y | S]\) as

\begin{equation}\protect\hypertarget{eq-glm-ch1}{}{
[Y | S] = \prod_{i=1}^n [Y_i | S(x_i)].
}\label{eq-glm-ch1}\end{equation}

The final step then consists of specifying a distribution for
\([Y_i | S(x_i)]\). Table~\ref{tbl-glm} gives the range, mean and
variance the three specifications for \${[}Y\_i \textbar{}
S(x\_i){]}\$\$ which we will consider in this book. In
Table~\ref{tbl-glm}, the \emph{canonical function}, say \(g(\cdot)\),
denotes the natural transformation of the mean component \(\mu_i\) that
allows us to introduce both covariates and the spatial process
\(S(x_i)\) into the model so as to explain the variation in \(\mu_i\) as

\begin{equation}\protect\hypertarget{eq-linear-predictor-ch1}{}{
g(\mu_i) = d(x_i)^\top \beta + S(x_i).
}\label{eq-linear-predictor-ch1}\end{equation}

where \(d(x_i)\) is a vector of spatially referenced covariates with
associated regression coefficients \(\beta\). Finally, the quantity
\(m_i\), which appears in the formulation of the Binomial and Poisson
distributions, is an offset quantity and is used to account for the
number of \emph{tests} or the population size at a given location
\(x_i\).

\hypertarget{tbl-glm}{}
\begin{longtable}[]{@{}
  >{\raggedright\arraybackslash}p{(\columnwidth - 8\tabcolsep) * \real{0.2000}}
  >{\raggedright\arraybackslash}p{(\columnwidth - 8\tabcolsep) * \real{0.2000}}
  >{\raggedright\arraybackslash}p{(\columnwidth - 8\tabcolsep) * \real{0.2000}}
  >{\raggedright\arraybackslash}p{(\columnwidth - 8\tabcolsep) * \real{0.2000}}
  >{\raggedright\arraybackslash}p{(\columnwidth - 8\tabcolsep) * \real{0.2000}}@{}}
\caption{\label{tbl-glm}Type of outcomes \(Y_{i}\) considered in this
book.}\tabularnewline
\toprule\noalign{}
\begin{minipage}[b]{\linewidth}\raggedright
Distribution
\end{minipage} & \begin{minipage}[b]{\linewidth}\raggedright
Range of \(Y_i\)
\end{minipage} & \begin{minipage}[b]{\linewidth}\raggedright
Mean of \([Y_i | S(x_i)]\)
\end{minipage} & \begin{minipage}[b]{\linewidth}\raggedright
Variance of \([Y_i | S(x_i)]\)
\end{minipage} & \begin{minipage}[b]{\linewidth}\raggedright
Canonical link
\end{minipage} \\
\midrule\noalign{}
\endfirsthead
\toprule\noalign{}
\begin{minipage}[b]{\linewidth}\raggedright
Distribution
\end{minipage} & \begin{minipage}[b]{\linewidth}\raggedright
Range of \(Y_i\)
\end{minipage} & \begin{minipage}[b]{\linewidth}\raggedright
Mean of \([Y_i | S(x_i)]\)
\end{minipage} & \begin{minipage}[b]{\linewidth}\raggedright
Variance of \([Y_i | S(x_i)]\)
\end{minipage} & \begin{minipage}[b]{\linewidth}\raggedright
Canonical link
\end{minipage} \\
\midrule\noalign{}
\endhead
\bottomrule\noalign{}
\endlastfoot
Gaussian & \((-\infty, +\infty)\) & \(\mu_i\) & \(\tau^2\) &
\(g(\mu_i) = \mu_i\) \\
Binomial & \(1,\dots,m_i\) & \(m_i\mu_i\) & \(m_i\mu_i(1-\mu_i)\) &
\(g(\mu_i) = \log\{ \mu_i/(1-\mu_i) \}\) \\
Poisson & \(1,2,\ldots,\infty\) & \(m_i\mu_i\) & \(m_i\mu_i\) &
\(g(\mu_i) = \log\{ \mu_i \}\) \\
\end{longtable}

Based on the formulation in (\ref{eq-linear-predictor-ch1}), we can see
that \(S(x_i)\) quantifies residual spatial effects on \(\mu_i\) that
have not been accounted for by the covariates \(d(x_i)\). In an ideal
scenario, the covariates \(d(x_i)\) should explain all the spatial
variation without the need for \(S(x_i)\). Although this unrealistic, in
practice we may be able to most of the variation in \(\mu_i\) through
\(d(x_i)\) and, hence, reduce \(S(x_i)\) to a negligible component. In
Chapter 2, we will show how a thorough exploratory analysis can help to
understand whether we have come close to that ideal scenario or, if
instead, we need the use of GLGM to model the data.

The model described in (\ref{eq-linear-predictor-ch1}) can be seen as
the most basic GLGM that can be used for a geostatistical analysis. As
we will see in the analysis of some of the examples and, in Chapter 6,
for the case studies, extensions of this model will be required to
accommodate the intrinsic non-spatial random variation of the data which
is not captured by the covariates.

The types of problems that statistical models are applied to can be
distinguished into three main categories: prediction problems;
explanatory problems; problems of hypothesis testing. Most of the times,
geostatistical problems tend to fall under the first category, where the
goal is make predictive inferences on the process \(S(x)\) at location
\(x\), which is usually outside of the set of sampled locations.
However, as will illustrate in the later chapters, geostatistical models
play an important also in the other two types of problems. In
particular, we will show that spatial correlation can have a substantial
impact on the point estimates and standard errors for \(\beta\). Hence,
if the goal of the analysis is explain the relationship between a
covariate \(d(x)\) with the mean component \(\mu\).

\hypertarget{workflow-of-a-statistical-analysis-and-structure-of-the-book}{%
\section{Workflow of a statistical analysis and structure of the
book}\label{workflow-of-a-statistical-analysis-and-structure-of-the-book}}

\begin{figure}

{\centering \includegraphics{./figures/workflow_diagram.png}

}

\caption{\label{fig-stages}Stages of a statistical analysis}

\end{figure}

Figure~\ref{fig-stages} shows the different stages that will follow in
carrying the geostatistical analysis of the examples introduced in
Section~\ref{sec-examples-ch1}. The exploratory analysis of the data is
an essential first step that is used to understand the empirical
associations between risk factors and the the health outcome of
interest. In our case, this first stage is also used to justify the use
of geostatistical models by questioning the underlying assumptions of
standard generalized linear models. Based on the results obtained from
the exploration of the data, we then formulate a suitable statistical
model and estimate its parameters using likelihood based methods of
inference. These also allows us to obtain uncertainty measures about the
strength of associations of regression relationships and the other model
parameters that define the shape of the spatial correlation in the data.
Following the estimation of the model, we then proceed to validate its
underlying assumptions using suitable diagnostics that assess whether
the model can later be sufficiently trusted to represent the observed
variation in the modelled outcome. At this stage, if the diagnostics
checks yield results that indicate the incompatibility of the model with
the data, we then back to the stage of model formulation and address the
issues arisen from the validation stage. If instead, we do not find any
evidence against the fitted model we can proceed to carry out sptatial
prediction. At this stage, it is important to define suitable predictive
targets that can help us to better answer the original research question
and better assist the decision making process. The final step of
visualization of uncertainty plays an important role in geostatistical
analysis in order to convey the main findings of the study in an
effective and easy-to-understand way for a wider audience which also
consists of non-experts.

In the remainder of this book, each chapter focuses on a specific stage
as shown in Figure~\ref{fig-stages}. We treat visualization of
uncertainty together with spatial prediction in
Chapter~\ref{sec-geo-prediction}.

Chapter~\ref{sec-handling-data} will provide an overview of how to
handle saptial data in R, in particular raster and shape files. The
skills learned in this chapter will be applied throughout the book, and
will especially be useful in Chapter~\ref{sec-geo-prediction} and
Chapter~\ref{sec-case-studies} for generating predictive maps of the
modelled outcome.

Chapter~\ref{sec-estimation} focuses on the model building process and
estimation of geostatistical models. This chapter will show how to carry
out initial exploratory analyses of the data to inform the formulation
of suitable geostatistical models and how these can be fitted using
maximum likelihood estimation methods.

Chapter~\ref{sec-validation} illustrated the use of methods that can be
used to validate the assumptions and calibration of statistical models.

Chapter~\ref{sec-geo-prediction} shows how geostatistical models can be
used to carry out spatial prediction of a health outcome of interest
both on a spatially continuous and spatially aggregated scales.

Finally, Chapter~\ref{sec-case-studies} presents the application of all
the methods illustrated in the previous chapters to three additional
data-sets. This chapter offers a summary of the content of book by
putting together all the stages in the geostatistical analyses for each
of the three case studies, and illustrates additional functionalities of
the \texttt{RiskMap} R package not covered in the previous chapters.

\bookmarksetup{startatroot}

\hypertarget{sec-handling-data}{%
\chapter{Handling of spatial data in R}\label{sec-handling-data}}

This is a book created from markdown and executable code.

See (\textbf{knuth84?}) for additional discussion of literate
programming.

\begin{Shaded}
\begin{Highlighting}[]
\DecValTok{1} \SpecialCharTok{+} \DecValTok{1}
\end{Highlighting}
\end{Shaded}

\begin{verbatim}
[1] 2
\end{verbatim}

\hypertarget{importing-and-processing-spatial-data-in-r}{%
\section{Importing and processing spatial data in
R}\label{importing-and-processing-spatial-data-in-r}}

\hypertarget{visualizing-geostatistical-data}{%
\section{Visualizing geostatistical
data}\label{visualizing-geostatistical-data}}

\hypertarget{section}{%
\section{}\label{section}}

\bookmarksetup{startatroot}

\hypertarget{sec-estimation}{%
\chapter{Model formulation and parameter
estimation}\label{sec-estimation}}

\hypertarget{list-of-the-main-functions-used-in-the-chapter}{%
\section*{List of the main functions used in the
chapter}\label{list-of-the-main-functions-used-in-the-chapter}}
\addcontentsline{toc}{section}{List of the main functions used in the
chapter}

\markright{List of the main functions used in the chapter}

\begin{longtable}[]{@{}
  >{\raggedright\arraybackslash}p{(\columnwidth - 4\tabcolsep) * \real{0.2222}}
  >{\raggedright\arraybackslash}p{(\columnwidth - 4\tabcolsep) * \real{0.2222}}
  >{\raggedright\arraybackslash}p{(\columnwidth - 4\tabcolsep) * \real{0.5556}}@{}}
\toprule\noalign{}
\begin{minipage}[b]{\linewidth}\raggedright
Function
\end{minipage} & \begin{minipage}[b]{\linewidth}\raggedright
R Package
\end{minipage} & \begin{minipage}[b]{\linewidth}\raggedright
Used for
\end{minipage} \\
\midrule\noalign{}
\endhead
\bottomrule\noalign{}
\endlastfoot
\texttt{lmer} & \texttt{lme4} & Fitting linear mixed models \\
\texttt{glmer} & \texttt{lme4} & Fitting generalized linear mixed
models \\
\texttt{glgm} & \texttt{RiskMap} & Fitting generalized linear mixed
models \\
\texttt{s\_variogram} & \texttt{RiskMap} & Computing the empirical
variogram and carrying out permutation test for spatial independence \\
\end{longtable}

\hypertarget{exploratory-analysis}{%
\section{Exploratory analysis}\label{exploratory-analysis}}

As illustrated in Figure~\ref{fig-stages}, exploratory analysis is the
first step that should be carried out in a statistical analysis. This
stage is essential to inform how covariates should be introduced in the
model and, in our case, whether the variation unexplained by those
covariates exhibits spatial correlation.

In the exploratory analysis of count data, we will also look at how
overdispersion, which is a necessary, though not sufficient, condition
for residual spatial correlation.

\hypertarget{sec-expl-assoc}{%
\subsection{Exploring associations with risk factors using count
data}\label{sec-expl-assoc}}

Assessment of the association between the health outcome of interest and
non-categorical (i.e.~continuous) risk factors can be carried using
graphical tools, such scatter plots. The graphical inspection of the
empirical association between the outcome and the covariates is
especially useful to identify non-linear patterns in the relationship
which should then be accounted for in the model formulation.

In this section, we look more closely at the case when the observed
outcome is a count which requires a different treatment from
continuously measured outcomes, which are generally covered by most
statistics textbooks (see, for example, Chapter 1 of Weisberg (2014)).

\hypertarget{when-the-outcome-is-an-aggregated-count}{%
\subsubsection{When the outcome is an aggregated
count}\label{when-the-outcome-is-an-aggregated-count}}

Let us first consider the example of the river-blindness data in Liberia
(Section~\ref{sec-rb-ch1}), and examine the association between
prevalence and elevation. We first generate a plot of the prevalence
against the measured elevation at each of the sample locations

\begin{verbatim}
Linking to GEOS 3.10.2, GDAL 3.4.3, PROJ 8.2.1; sf_use_s2() is TRUE
\end{verbatim}

\begin{Shaded}
\begin{Highlighting}[]
\NormalTok{liberia}\SpecialCharTok{$}\NormalTok{prev }\OtherTok{\textless{}{-}}\NormalTok{ liberia}\SpecialCharTok{$}\NormalTok{npos}\SpecialCharTok{/}\NormalTok{liberia}\SpecialCharTok{$}\NormalTok{ntest}

\FunctionTok{ggplot}\NormalTok{(liberia, }\FunctionTok{aes}\NormalTok{(}\AttributeTok{x =}\NormalTok{ elevation, }\AttributeTok{y =}\NormalTok{ prev)) }\SpecialCharTok{+} \FunctionTok{geom\_point}\NormalTok{() }\SpecialCharTok{+}
  \FunctionTok{labs}\NormalTok{(}\AttributeTok{x=}\StringTok{"Elevation (meters)"}\NormalTok{,}\AttributeTok{y=}\StringTok{"Prevalence"}\NormalTok{)}
\end{Highlighting}
\end{Shaded}

\begin{figure}[H]

{\centering \includegraphics{03_model-fitting_files/figure-pdf/fig-prev-elev-liberia-1.pdf}

}

\caption{\label{fig-prev-elev-liberia}Scatter plot of the empirical
prevalence for river-blindess against elevation, measured in meters.}

\end{figure}

The plot shown in Figure~\ref{fig-prev-elev-liberia} shows that, as
elevation increases from 0 to around 150 meters, prevalence rapidly
increases to around 0.25 and, for larger values in elevation than 150
meters, the relationship levels off. This begs the question of how we
can account for this in a regression model. To answer this question
rigorously, however, the plot in Figure~\ref{fig-prev-elev-liberia}
cannot be used. This is because, when the modelled outcome is a bounded
Binomial count, regression relationships are specified on the
logit-transformed prevalence (log-odds) scale; see Table~\ref{tbl-glm}
in Section Section~\ref{sec-geostat-models} . To explore regression
relationships in the case of prevalence data, it is convenient to use
the so-called empirical logit in place of the empirical prevalence. The
empirical logit is defined as

\begin{equation}\protect\hypertarget{eq-empirical-logit}{}{
l_{i} = \log\left\{\frac{y_i + 1/2}{n_i - y_i + 1/2}\right\}
}\label{eq-empirical-logit}\end{equation}

where \(y_i\) are the number of individuals who tested positive for
riverblindness and \(n_i\) is the total number of people tested at a
location. The reason for using the empirical logit, rather than the
standard logit transformation applied directly to the empirical
prevalence, is that it allows to generate finite values for empirical
prevalence values of 0 and 1, for which the standard logit
transformation is not defined.

\begin{Shaded}
\begin{Highlighting}[]
\CommentTok{\# The empirical logit}
\NormalTok{liberia}\SpecialCharTok{$}\NormalTok{elogit }\OtherTok{\textless{}{-}} \FunctionTok{log}\NormalTok{((liberia}\SpecialCharTok{$}\NormalTok{npos}\FloatTok{+0.5}\NormalTok{)}\SpecialCharTok{/}
\NormalTok{                      (liberia}\SpecialCharTok{$}\NormalTok{ntest}\SpecialCharTok{{-}}\NormalTok{liberia}\SpecialCharTok{$}\NormalTok{npos}\FloatTok{+0.5}\NormalTok{))}

\FunctionTok{ggplot}\NormalTok{(liberia, }\FunctionTok{aes}\NormalTok{(}\AttributeTok{x =}\NormalTok{ elevation, }\AttributeTok{y =}\NormalTok{ elogit)) }\SpecialCharTok{+} \FunctionTok{geom\_point}\NormalTok{() }\SpecialCharTok{+}
  
  \CommentTok{\# Adding a smoothing spline}
  \FunctionTok{labs}\NormalTok{(}\AttributeTok{x=}\StringTok{"Elevation (meters)"}\NormalTok{,}\AttributeTok{y=}\StringTok{"Empirical logit"}\NormalTok{) }\SpecialCharTok{+}
  \FunctionTok{stat\_smooth}\NormalTok{(}\AttributeTok{method =} \StringTok{"gam"}\NormalTok{, }\AttributeTok{formula =}\NormalTok{ y }\SpecialCharTok{\textasciitilde{}} \FunctionTok{s}\NormalTok{(x),}\AttributeTok{se=}\ConstantTok{FALSE}\NormalTok{)}\SpecialCharTok{+}
  
  \CommentTok{\# Adding linear regression fit with log{-}transformed elevation}
  \FunctionTok{stat\_smooth}\NormalTok{(}\AttributeTok{method =} \StringTok{"lm"}\NormalTok{, }\AttributeTok{formula =}\NormalTok{ y }\SpecialCharTok{\textasciitilde{}} \FunctionTok{log}\NormalTok{(x),}
              \AttributeTok{col=}\StringTok{"green"}\NormalTok{,}\AttributeTok{lty=}\StringTok{"dashed"}\NormalTok{,}\AttributeTok{se=}\ConstantTok{FALSE}\NormalTok{) }\SpecialCharTok{+}

  \CommentTok{\# Adding linear regression fit with change point in 150 meters}
  \FunctionTok{stat\_smooth}\NormalTok{(}\AttributeTok{method =} \StringTok{"lm"}\NormalTok{, }\AttributeTok{formula =}\NormalTok{ y }\SpecialCharTok{\textasciitilde{}}\NormalTok{ x }\SpecialCharTok{+} \FunctionTok{pmax}\NormalTok{(x}\DecValTok{{-}150}\NormalTok{, }\DecValTok{0}\NormalTok{),}
              \AttributeTok{col=}\StringTok{"red"}\NormalTok{,}\AttributeTok{lty=}\StringTok{"dashed"}\NormalTok{,}\AttributeTok{se=}\ConstantTok{FALSE}\NormalTok{) }
  
\end{Highlighting}
\end{Shaded}

\begin{figure}[H]

{\centering \includegraphics{03_model-fitting_files/figure-pdf/fig-elogit-elev-liberia-1.pdf}

}

\caption{\label{fig-elogit-elev-liberia}Scatter plot of the empirical
prevalence for river-blindess against elevation, measured in meters.}

\end{figure}

Figure~\ref{fig-elogit-elev-liberia} shows the scatter plot of the
empirical logit against elevation. In this plot, we have also added
three lines though the \texttt{stat\_smooth} from the \texttt{ggplot2}
package. Using this function, we first pass the term \texttt{gam} to
\texttt{method} to add a penalized smoothing spline (Hastie, Tibshirani,
and Friedman 2001), represented by the blue solid line. The smoothing
spline allows us to better discern how the type of relationship and how
to best capture it using a standard regression approach. As er can see
from Figure~\ref{fig-elogit-elev-liberia}, the smoothing spline
corroborates our initial observation of a positive relationship up to
about 150 meters, followed by a plateau.

To capture this non-linear relationship, we can use the two following
approaches. The first is based on a simple log-transformation of
elevation and is represented in Figure~\ref{fig-elogit-elev-liberia} by
the green line. If were to express this relationship using a standard
Binomial regression model, this would take the form
\begin{equation}\protect\hypertarget{eq-log-elev-glm}{}{
\log\left\{\frac{p(x_i)}{1-p(x_i)}\right\} = \beta_0 + \beta_1 \log\{e(x_i)\}
}\label{eq-log-elev-glm}\end{equation} where \(p(x_i)\) and \(e(x_i)\)
are the river-blindness prevalence and elevation at sampled location
\(x_i\), respectively.

Alternatively, the non-linear effect of elevation on prevalence could be
captured using a linear spline. Put in simple terms, we want to fit a
linear regression model that allows for a change in slope above 150
meters. Formally, this is expressed in a Binomial regression model as
\begin{equation}\protect\hypertarget{eq-lspline-elev-glm}{}{
\log\left\{\frac{p(x_i)}{1-p(x_i)}\right\} = \beta_0 + \beta_1 e(x_i) + \beta_{2} \max\{e(x_i)-150, 0\}.
}\label{eq-lspline-elev-glm}\end{equation} Based on the equation above,
the effect of elevation below 150 meters is quantified by the parameter
\(\beta_1\). Above 150 meters, instead, the effect of elevation becomes
\(\beta_1 + \beta_2\). Note that the function \texttt{pmax} (and not the
standard base function \texttt{max}) should be used in R when the
computation of the maximum between a scalar value and each of the
components of a numeric vector is required.

Before proceeding further, it is important to explain the differences
between the use of the logarithmic transformation
(Equation~\ref{eq-log-elev-glm}) and the linear spline
(Equation~\ref{eq-lspline-elev-glm}). We observe that both curves
provide a similar fit to the data, with larger differences observed for
larger values in elevation, where the log-transformed elevation models
yields larger values for the predicted prevalence. This also suggests
that if we were to extrapolate the predictions beyond 600 meters in
elevation the implied pattern by the model with the log-transformed
elevation would predict an increasingly larger elevation, which is
unrealistic, since the fly that transmits the diseases cannot breed at
those altitudes. The linear spline model instead would generate
predictions that would be very similar to those observed between 150 and
600 meters. From this point view, the linear spline model would thus
have more scientific validity than the other model. However, which of
the two approaches should be chosen to model the effect of elevation is
a question that closely depends on the research question to be
addressed.

If the interest of the study was in better understanding the association
between elevation and prevalence, the linear spline model does not only
provide a more credible explanation but also its regression parameters
can be more easily interpreted. In fact, for a unit increase in
elevation, the multiplicative change in the odds for river-blindness is
\(\exp\{\beta_1\}\), if elevation is below 150 meters, and
\(\exp\{\beta_1+\beta_2\}\), if elevation is above 150 meters. When
instead we use the log-transformed elevation, the interpretation of
\(\beta_1\) in Equation~\ref{eq-log-elev-glm} is slightly more
complicated, as it is based on the multiplicative increase in elevation
by the same amount given by the base of the algorithm, which is about
\(e \approx 2.718\)\footnote{The letter \(e\) stands for the so called
  Euler's number and represents the base of the natural logarithm. In
  the book, we write \(\log(\cdot)\) to mean the ``natural logarithm of
  \(\cdot\)''.}. To avoid this, one could rescale the regression
coefficient as, for example, \(\beta_1/\log_{2}(e)\) which would be
interpreted as the multiplicative change in the odds for river-blindness
for a doubling in elevation. However, a doubling in elevation is less
meaningful when considering larger values of elevation.

When the goal of statistical analysis is instead in developing a
predictive model for the outcome of interest, the explanatory power and
interpretability of the model may be of less concern. For this reason,
the model with the log-transformed model could be preferred over the
model with the linear spline, if it shown to yield more predictive
power. We will come back to this point again in
Chapter~\ref{sec-geo-prediction}, where will show how to assess and
compare the predictive performance of different geostatistical models.

The other type of aggregated count data that we consider are unbounded
counts. The Anopheles mosquitoes data-set
(Section~\ref{sec-mosq-data-ch1}) is an example of this, since there is
no upper limit to the number of mosquitoes that can be trapped at a
location. Let us consider the covariate represented by elevation. In
this case, the simplest model that can be used to analyse the data is a
Poisson regression, where the linear predictor is defined on the log of
the mean number of mosquitoes (Table~\ref{tbl-glm}). Hence, exploratory
plots for the association with covariates should be generated using the
log transformed counts of mosquitoes. In this instance, to avoid taking
the log of zero, we can add 1 to the reported counts, if required. The
variable of the \texttt{An.gambiae} in the \texttt{anopheles} data-set
does not contain any 0, hence we simply apply the log tranformation
without adding 1.

\begin{Shaded}
\begin{Highlighting}[]
\NormalTok{anopheles}\SpecialCharTok{$}\NormalTok{log\_counts }\OtherTok{\textless{}{-}} \FunctionTok{log}\NormalTok{(anopheles}\SpecialCharTok{$}\NormalTok{An.gambiae)}
\FunctionTok{ggplot}\NormalTok{(anopheles, }\FunctionTok{aes}\NormalTok{(}\AttributeTok{x =}\NormalTok{ elevation, }\AttributeTok{y =}\NormalTok{ log\_counts)) }\SpecialCharTok{+} \FunctionTok{geom\_point}\NormalTok{() }\SpecialCharTok{+}
  
  \CommentTok{\# Adding a smoothing spline}
  \FunctionTok{labs}\NormalTok{(}\AttributeTok{x=}\StringTok{"Elevation (meters)"}\NormalTok{,}\AttributeTok{y=}\StringTok{"Log number of An. gambiae mosquitoes"}\NormalTok{) }\SpecialCharTok{+}
  \FunctionTok{stat\_smooth}\NormalTok{(}\AttributeTok{method =} \StringTok{"lm"}\NormalTok{, }\AttributeTok{formula =}\NormalTok{ y }\SpecialCharTok{\textasciitilde{}}\NormalTok{ x, }\AttributeTok{se=}\ConstantTok{FALSE}\NormalTok{)}
\end{Highlighting}
\end{Shaded}

\begin{figure}[H]

{\centering \includegraphics{03_model-fitting_files/figure-pdf/fig-logcounts-elev-mosq-1.pdf}

}

\caption{\label{fig-logcounts-elev-mosq}Scatter plot of the log
tranformed number of \emph{Anopheles gambiae} mosquitoes against
elevation, measured in meters. The blue line is generated using the
least squares fit.}

\end{figure}

The scatter plot of Figure~\ref{fig-logcounts-elev-mosq} shows that
there is a negative, though weak, association, with the average number
of mosquitoes decreasing for increasing elevation. In this instance, the
assumption of a linear relationship with elevation would be a reasonable
choice.

\hypertarget{when-the-outcome-is-an-invidual-level-binary-indicator}{%
\subsubsection{When the outcome is an invidual-level binary
indicator}\label{when-the-outcome-is-an-invidual-level-binary-indicator}}

We now consider the malaria data from Kenya
(Section~\ref{sec-malaria-ch1}) where the main outcome is the result
from a rapid diagnostic test (RDT) for malaria from individuals within
households. In this case, because the outcome only takes two values, 1
for a positive RDT test result and 0 otherwise, the direct application
of the empirical logit from Equation~\ref{eq-empirical-logit} would not
help us to generate informative scatter plots. Throughout the book, we
will consider the data from the community survey only, hence we work
with a subset of the data which we shall name \texttt{malkenya\_comm}

\begin{Shaded}
\begin{Highlighting}[]
\NormalTok{malkenya\_comm }\OtherTok{\textless{}{-}}\NormalTok{ malkenya[malkenya}\SpecialCharTok{$}\NormalTok{Survey}\SpecialCharTok{==}\StringTok{"community"}\NormalTok{, ]}
\end{Highlighting}
\end{Shaded}

To show how this issue can be overcome, let us consider the variables
age and gender. To generate a plot that can help us understand between
the relationship with malaria prevalence and the two risk factors, we
proceed as follows.

\begin{Shaded}
\begin{Highlighting}[]
\CommentTok{\# Grouping of ages into classes defined through "breaks"}
\NormalTok{malkenya\_comm}\SpecialCharTok{$}\NormalTok{Age\_class }\OtherTok{\textless{}{-}} \FunctionTok{cut}\NormalTok{(malkenya\_comm}\SpecialCharTok{$}\NormalTok{Age, }
                            \AttributeTok{breaks =} \FunctionTok{c}\NormalTok{(}\DecValTok{0}\NormalTok{, }\DecValTok{5}\NormalTok{, }\DecValTok{10}\NormalTok{, }\DecValTok{15}\NormalTok{, }\DecValTok{30}\NormalTok{, }\DecValTok{40}\NormalTok{, }\DecValTok{50}\NormalTok{, }\DecValTok{100}\NormalTok{),}
                            \AttributeTok{include.lowest =} \ConstantTok{TRUE}\NormalTok{)}
\end{Highlighting}
\end{Shaded}

Using the \texttt{cut} function, we first split age (in years) into
classes through the argument \texttt{breaks}. The classification of age
into \([0,5]\), \((5, 10]\) and \((10, 15]\) is common in many malaria
epidemiology studies, as children are one of the groups at highest risk
malaria. The choice of the other classes of age reflects instead the
need to balance the number of observations falling in each of the
classes.

\begin{Shaded}
\begin{Highlighting}[]
\CommentTok{\# Computation of the empirical logit by age groups and gender}
\NormalTok{age\_class\_data }\OtherTok{\textless{}{-}} \FunctionTok{aggregate}\NormalTok{(RDT }\SpecialCharTok{\textasciitilde{}}\NormalTok{ Age\_class }\SpecialCharTok{+}\NormalTok{ Gender, }
                                    \AttributeTok{data =}\NormalTok{ malkenya\_comm, }
                                    \AttributeTok{FUN =} \ControlFlowTok{function}\NormalTok{(y) }
                                    \FunctionTok{log}\NormalTok{((}\FunctionTok{sum}\NormalTok{(y)}\SpecialCharTok{+}\FloatTok{0.5}\NormalTok{)}\SpecialCharTok{/}\NormalTok{(}\FunctionTok{length}\NormalTok{(y)}\SpecialCharTok{{-}}\FunctionTok{sum}\NormalTok{(y)}\SpecialCharTok{+}\FloatTok{0.5}\NormalTok{)))}
\end{Highlighting}
\end{Shaded}

We then compute the empirical logit, using the total number of cases
within age group and by gender. For a given age group and gender, which
we denote as \(\mathcal{C}\), the empirical logit in
Equation~\ref{eq-empirical-logit}, now takes the form
\begin{equation}\protect\hypertarget{eq-empirical-logit-bin}{}{
l_{\mathcal{C}} = \log\left\{\frac{\sum_{i \in \mathcal{C}} y_{i} + 0.5}{|\mathcal{C}|- \sum_{i \in \mathcal{C}} y_{i} + 0.5} \right\}
}\label{eq-empirical-logit-bin}\end{equation} where \(y_i\) are the
individual binary outcomes and \(i\in \mathcal{C}\) is used to indicate
that the sum is carried out over all the individuals who belong the
class \(\mathcal{C}\), identified by a specific age group and gender.
Finally, \(|\mathcal{C}|\) is the number of individuals who fall within
\(\mathcal{C}\). In the code above, the empirical logit in
Equation~\ref{eq-empirical-logit-bin} is computed using the
\texttt{aggregate} function. An inspection of the object
\texttt{age\_class\_data}, a data frame, shows that the empirical is
found in the column named \texttt{RDT}.

\begin{Shaded}
\begin{Highlighting}[]
\CommentTok{\# Computation of the average age within each age group }
\NormalTok{age\_class\_data}\SpecialCharTok{$}\NormalTok{age\_mean\_point }\OtherTok{\textless{}{-}} \FunctionTok{aggregate}\NormalTok{(Age }\SpecialCharTok{\textasciitilde{}}\NormalTok{ Age\_class }\SpecialCharTok{+}\NormalTok{ Gender, }
                                 \AttributeTok{data =}\NormalTok{ malkenya\_comm, }
                                 \AttributeTok{FUN =}\NormalTok{ mean)}\SpecialCharTok{$}\NormalTok{Age}


\CommentTok{\# Number of individuals within each age group, by gender}
\NormalTok{age\_class\_data}\SpecialCharTok{$}\NormalTok{n\_obs }\OtherTok{\textless{}{-}}  \FunctionTok{aggregate}\NormalTok{(Age }\SpecialCharTok{\textasciitilde{}}\NormalTok{ Age\_class }\SpecialCharTok{+}\NormalTok{ Gender, }
                         \AttributeTok{data =}\NormalTok{ malkenya\_comm, }
                         \AttributeTok{FUN =}\NormalTok{ length)}\SpecialCharTok{$}\NormalTok{Age}
\end{Highlighting}
\end{Shaded}

In order to generate the scatter-plot, we compute the average age within
each age group by gender, and use these as our values for the x-axis.
Note that since we only need to obtain the average age from this output,
we use \texttt{\$Age} to extract this only and allocate to the column
\texttt{age\_mean\_point}. Finally, we also compute the number of
observations within each of classes and place this in \texttt{n\_obs}.

\begin{Shaded}
\begin{Highlighting}[]
\FunctionTok{ggplot}\NormalTok{(age\_class\_data, }\FunctionTok{aes}\NormalTok{(}\AttributeTok{x =}\NormalTok{ age\_mean\_point, }\AttributeTok{y =}\NormalTok{ RDT, }
                           \AttributeTok{size =}\NormalTok{ n\_obs, }
                           \AttributeTok{colour =}\NormalTok{ Gender)) }\SpecialCharTok{+} 
  \FunctionTok{geom\_point}\NormalTok{() }\SpecialCharTok{+} 
  \FunctionTok{labs}\NormalTok{(}\AttributeTok{x=}\StringTok{"Age (years)"}\NormalTok{,}\AttributeTok{y=}\StringTok{"Empirical logit"}\NormalTok{)  }
\end{Highlighting}
\end{Shaded}

\begin{figure}[H]

{\centering \includegraphics{03_model-fitting_files/figure-pdf/fig-elogit-age-malkenya-1.pdf}

}

\caption{\label{fig-elogit-age-malkenya}Plot of the empirical logit
against age, for males and females. The size of each solid point is
rendered proportional to the number of individuals within age group, as
indicated in the legend.}

\end{figure}

The resulting plot in Figure~\ref{fig-elogit-age-malkenya} shows the
empirical logit against age by gender, with the size of each of the
points proportional to the number of observations falling within each
class. The observed patterns are explained by the fact that young
children, especially those under the age of five, are particularly
vulnerable to severe malaria infections. This is primarily due to their
immature immune systems and lack of acquired immunity. As individuals
grow older, they generally develop partial immunity to malaria through
repeated exposure to the disease. This acquired immunity can provide
some level of protection against severe malaria. At the same time,
gender roles and activities can influence exposure to malaria-carrying
mosquitoes. For example, men may spend more time outdoors for work or
other activities, increasing their exposure to mosquito bites and thus
their risk of infection. In addition, there are also biological factors
to consider. Hormonal and genetic differences between males and females
may also contribute to variations in immune responses to malaria
infection. The interaction between age and gender is complex and may
vary depending on the specific context and population being studied. A
2020 report from the Bill \& Melinda Gates foundation provides a
detailed overview of this and other aspects related to gender and
malaria (Katz and Bill \& Melinda Gates Foundation 2020).

To account for age in a model for malaria prevalence, several approaches
are possible, some of which have been developed using biological models
(Smith et al. 2007). To model the patterns observed in
Figure~\ref{fig-elogit-age-malkenya}, we can follow the same approach
used in the previous section to model the relationship between elevation
and river-blindness prevalence. First, let us consider age without the
effect of gender. Let \(p_{j}(x_i)\) denote the probability of a
positive RDT for the \(j\)-th individual living in a household at
location \(x_i\). Assuming that malaria risk reaches its peak at 15
years of age, we can capture the non-linear relationship using a linear
spline with two knots, one at 15 years and a second one at 40 years.
This is expressed as
\begin{equation}\protect\hypertarget{eq-age-only-reg}{}{
\begin{aligned}
\log\left\{\frac{p_{j}(x_i)}{1-p_j(x_i)}\right\} = \beta_{0} + \beta_{1}a_{ij}+\beta_{2} \times\max\{a_{ij}-15, 0\} + \beta_{3}\max\{a_{ij}-40, 0\}
\end{aligned}
}\label{eq-age-only-reg}\end{equation} where \(a_{ij}\) is the age, in
years, for the \(j\)-th individual at household \(i\). Based on this
model the effect of age on RDT prevalence is \(\beta_{1}\), for
\(a_{ij} < 15\), \(\beta_{1}+\beta_{2}\), for \(15 < a_{ij} < 40\), and
\(\beta_{1}+\beta_{2}+\beta_{3}\) for \(a_{ij} > 40\).

Figure~\ref{fig-elogit-age-malkenya} indicates that age may interact
with gender, meaning that the effect of gender on RDT prevalence changes
across age, with larger differences observed between males and females
for ages above 20 years. To assess such differences using a standard
Binomial regression model, the linear predictor for RDT prevalence can
be formulated as
\begin{equation}\protect\hypertarget{eq-age-inter-reg}{}{
\begin{aligned}
\log\left\{\frac{p_{j}(x_i)}{1-p_j(x_i)}\right\} = \beta_{0} + (\beta_{1} + \beta_{1}^*g_{ij})\times a_{ij}+(\beta_{2} + \beta_{2}^*g_{ij})\times\max\{a_{ij}-15, 0\} + \\
(\beta_{3} + \beta_{3}^*g_{ij}) \times \max\{a_{ij}-40, 0\}
\end{aligned}
}\label{eq-age-inter-reg}\end{equation} where \(g_{ij}\) is the
indicator for gender, with 1 corresponding to male and 0 to female. The
coefficients \(\beta_{1}^*\), \(\beta_{2}^*\) and \(\beta_{3}^*\) thus
quantify the differences in risk between the two genders for ages below
15 years, betwee 15 and 40 years, and above 40 years, respectively. If
all of those coefficients were 0, the model in
Equation~\ref{eq-age-only-reg} would be recovered.

\begin{Shaded}
\begin{Highlighting}[]
\NormalTok{glm\_age\_gender\_interaction }\OtherTok{\textless{}{-}} \FunctionTok{glm}\NormalTok{(RDT }\SpecialCharTok{\textasciitilde{}}\NormalTok{ Age }\SpecialCharTok{+}\NormalTok{ Gender}\SpecialCharTok{:}\NormalTok{Age }\SpecialCharTok{+} 
                                  \FunctionTok{pmax}\NormalTok{(Age}\DecValTok{{-}15}\NormalTok{, }\DecValTok{0}\NormalTok{) }\SpecialCharTok{+}\NormalTok{ Gender}\SpecialCharTok{:}\FunctionTok{pmax}\NormalTok{(Age}\DecValTok{{-}15}\NormalTok{, }\DecValTok{0}\NormalTok{) }\SpecialCharTok{+} 
                                  \FunctionTok{pmax}\NormalTok{(Age}\DecValTok{{-}40}\NormalTok{, }\DecValTok{0}\NormalTok{) }\SpecialCharTok{+}\NormalTok{ Gender}\SpecialCharTok{:}\FunctionTok{pmax}\NormalTok{(Age}\DecValTok{{-}40}\NormalTok{, }\DecValTok{0}\NormalTok{),}
                              \AttributeTok{data =}\NormalTok{ malkenya\_comm, }\AttributeTok{family =}\NormalTok{ binomial)}

\FunctionTok{summary}\NormalTok{(glm\_age\_gender\_interaction)}
\DocumentationTok{\#\# }
\DocumentationTok{\#\# Call:}
\DocumentationTok{\#\# glm(formula = RDT \textasciitilde{} Age + Gender:Age + pmax(Age {-} 15, 0) + Gender:pmax(Age {-} }
\DocumentationTok{\#\#     15, 0) + pmax(Age {-} 40, 0) + Gender:pmax(Age {-} 40, 0), family = binomial, }
\DocumentationTok{\#\#     data = malkenya\_comm)}
\DocumentationTok{\#\# }
\DocumentationTok{\#\# Deviance Residuals: }
\DocumentationTok{\#\#     Min       1Q   Median       3Q      Max  }
\DocumentationTok{\#\# {-}0.7681  {-}0.7051  {-}0.4940  {-}0.2734   2.7294  }
\DocumentationTok{\#\# }
\DocumentationTok{\#\# Coefficients:}
\DocumentationTok{\#\#                              Estimate Std. Error z value Pr(\textgreater{}|z|)    }
\DocumentationTok{\#\# (Intercept)                  {-}1.05835    0.10245 {-}10.331  \textless{} 2e{-}16 ***}
\DocumentationTok{\#\# Age                          {-}0.03384    0.01310  {-}2.584  0.00978 ** }
\DocumentationTok{\#\# pmax(Age {-} 15, 0)            {-}0.03975    0.02356  {-}1.687  0.09162 .  }
\DocumentationTok{\#\# pmax(Age {-} 40, 0)             0.09170    0.02482   3.695  0.00022 ***}
\DocumentationTok{\#\# Age:GenderMale                0.01428    0.01221   1.170  0.24202    }
\DocumentationTok{\#\# GenderMale:pmax(Age {-} 15, 0) {-}0.03625    0.03145  {-}1.153  0.24908    }
\DocumentationTok{\#\# GenderMale:pmax(Age {-} 40, 0)  0.02451    0.04320   0.567  0.57052    }
\DocumentationTok{\#\# {-}{-}{-}}
\DocumentationTok{\#\# Signif. codes:  0 \textquotesingle{}***\textquotesingle{} 0.001 \textquotesingle{}**\textquotesingle{} 0.01 \textquotesingle{}*\textquotesingle{} 0.05 \textquotesingle{}.\textquotesingle{} 0.1 \textquotesingle{} \textquotesingle{} 1}
\DocumentationTok{\#\# }
\DocumentationTok{\#\# (Dispersion parameter for binomial family taken to be 1)}
\DocumentationTok{\#\# }
\DocumentationTok{\#\#     Null deviance: 2875.8  on 3351  degrees of freedom}
\DocumentationTok{\#\# Residual deviance: 2673.8  on 3345  degrees of freedom}
\DocumentationTok{\#\# AIC: 2687.8}
\DocumentationTok{\#\# }
\DocumentationTok{\#\# Number of Fisher Scoring iterations: 5}
\end{Highlighting}
\end{Shaded}

The code above shows how to fit the model specified in
Equation~\ref{eq-age-inter-reg}. The terms \texttt{Age},
\texttt{pmax(Age-15,\ 0)} and \texttt{pmax(Age-40,\ 0)} respectively
correspond to \(\beta_{1}\), \(\beta_{2}\) and \(\beta_{3}\), whilst the
\texttt{Gender:Age}, \texttt{Gender:pmax(Age-15,\ 0)} and
\texttt{Gender:pmax(Age-40,\ 0)} to \(\beta_{1}^*\), \(\beta_{2}^*\) and
\(\beta_{3}^*\), respectively. In the summary of the fitted model, we
observe that the interaction coefficients are non-statistically
significant. However, removing the interaction based on the fact that
each of the coefficients have each p-values larger than the conventional
level of 5\% would be wrong. Instead we should carry out the likelihood
ratio test, as shown below.

\begin{Shaded}
\begin{Highlighting}[]
\NormalTok{glm\_age\_gender\_no\_interaction }\OtherTok{\textless{}{-}} \FunctionTok{glm}\NormalTok{(RDT }\SpecialCharTok{\textasciitilde{}}\NormalTok{ Age }\SpecialCharTok{+}  \FunctionTok{pmax}\NormalTok{(Age}\DecValTok{{-}15}\NormalTok{, }\DecValTok{0}\NormalTok{) }\SpecialCharTok{+} \FunctionTok{pmax}\NormalTok{(Age}\DecValTok{{-}40}\NormalTok{, }\DecValTok{0}\NormalTok{),}
                              \AttributeTok{data =}\NormalTok{ malkenya\_comm, }\AttributeTok{family =}\NormalTok{ binomial)}

\FunctionTok{anova}\NormalTok{(glm\_age\_gender\_no\_interaction, glm\_age\_gender\_interaction, }\AttributeTok{test =} \StringTok{"Chisq"}\NormalTok{)}
\DocumentationTok{\#\# Analysis of Deviance Table}
\DocumentationTok{\#\# }
\DocumentationTok{\#\# Model 1: RDT \textasciitilde{} Age + pmax(Age {-} 15, 0) + pmax(Age {-} 40, 0)}
\DocumentationTok{\#\# Model 2: RDT \textasciitilde{} Age + Gender:Age + pmax(Age {-} 15, 0) + Gender:pmax(Age {-} }
\DocumentationTok{\#\#     15, 0) + pmax(Age {-} 40, 0) + Gender:pmax(Age {-} 40, 0)}
\DocumentationTok{\#\#   Resid. Df Resid. Dev Df Deviance Pr(\textgreater{}Chi)}
\DocumentationTok{\#\# 1      3348     2675.6                     }
\DocumentationTok{\#\# 2      3345     2673.8  3   1.8051   0.6138}
\end{Highlighting}
\end{Shaded}

To carry out the likelihood ratio test to assess the null hypothesis
that \(\beta_{1}^*=\beta_{2}^*=\beta_{3}^*=0\), we first fit the
simplified nested model under this null hypothesis. The likelihood ratio
test can then be carried out using the \texttt{anova} command as shown.
The p-value indicates that the we do not find evidence against the null
hypothesis, hence in our analysis of the data we might favour the
simplified model that does not assumes an interaction between the two
genders.

The approach just illustrated, can also be applied to explore the
association with other continuous variables that are a property of the
household and not of the individual. Let us, for example, consider the
variable \texttt{elevation} from the \texttt{malkenya} data-set.

\begin{Shaded}
\begin{Highlighting}[]
\NormalTok{malekenya\_comm }\OtherTok{\textless{}{-}}\NormalTok{ malkenya[malkenya}\SpecialCharTok{$}\NormalTok{Survey}\SpecialCharTok{==}\StringTok{"community"}\NormalTok{, ]}

\NormalTok{malkenya\_comm}\SpecialCharTok{$}\NormalTok{elevation\_class }\OtherTok{\textless{}{-}} \FunctionTok{cut}\NormalTok{(malkenya\_comm}\SpecialCharTok{$}\NormalTok{elevation,}
                            \AttributeTok{breaks =} \FunctionTok{quantile}\NormalTok{(malkenya\_comm}\SpecialCharTok{$}\NormalTok{elevation, }\FunctionTok{seq}\NormalTok{(}\DecValTok{0}\NormalTok{, }\DecValTok{1}\NormalTok{, }\AttributeTok{by =} \FloatTok{0.1}\NormalTok{)),}
                            \AttributeTok{include.lowest =} \ConstantTok{TRUE}\NormalTok{)}
\end{Highlighting}
\end{Shaded}

Following the same approach used for age, we first split elevation into
classes. To define these, we use the deciles of the empirical
distribution of \texttt{elevation} which we calculate using the
\texttt{quantile} function above. In this way we also ensure that the
number of observations falling within each class of elevation is
approximately the same.

\begin{Shaded}
\begin{Highlighting}[]
\CommentTok{\# Computation of the empirical logit by classes of elevation}
\NormalTok{elev\_class\_data }\OtherTok{\textless{}{-}} \FunctionTok{aggregate}\NormalTok{(RDT }\SpecialCharTok{\textasciitilde{}}\NormalTok{ elevation\_class, }
                                    \AttributeTok{data =}\NormalTok{ malkenya\_comm, }
                                    \AttributeTok{FUN =} \ControlFlowTok{function}\NormalTok{(y) }
                                    \FunctionTok{log}\NormalTok{((}\FunctionTok{sum}\NormalTok{(y)}\SpecialCharTok{+}\FloatTok{0.5}\NormalTok{)}\SpecialCharTok{/}\NormalTok{(}\FunctionTok{length}\NormalTok{(y)}\SpecialCharTok{{-}}\FunctionTok{sum}\NormalTok{(y)}\SpecialCharTok{+}\FloatTok{0.5}\NormalTok{)))}


\CommentTok{\# Computation of the average elevation within each class of elevation}
\NormalTok{elev\_class\_data}\SpecialCharTok{$}\NormalTok{elevation\_mean }\OtherTok{\textless{}{-}} \FunctionTok{aggregate}\NormalTok{(elevation }\SpecialCharTok{\textasciitilde{}}\NormalTok{ elevation\_class, }
                                    \AttributeTok{data =}\NormalTok{ malkenya\_comm, }
                                    \AttributeTok{FUN =}\NormalTok{ mean)}\SpecialCharTok{$}\NormalTok{elevation}
\end{Highlighting}
\end{Shaded}

We then compute the empirical logit and the average elevation for each
class of elevation. The empirical logit is computed as already defined
in Equation~\ref{eq-empirical-logit-bin}, where now the definition of
\(\mathcal{C}\) is given by a specific decile used to split the
distribution of elevation.

\begin{Shaded}
\begin{Highlighting}[]
\FunctionTok{ggplot}\NormalTok{(elev\_class\_data, }\FunctionTok{aes}\NormalTok{(}\AttributeTok{x =}\NormalTok{ elevation\_mean, }\AttributeTok{y =}\NormalTok{ RDT), }
                           \AttributeTok{size =}\NormalTok{ n\_obs) }\SpecialCharTok{+} 
  \FunctionTok{geom\_point}\NormalTok{() }\SpecialCharTok{+} 
  \FunctionTok{labs}\NormalTok{(}\AttributeTok{x=}\StringTok{"Elevation (meters)"}\NormalTok{,}\AttributeTok{y=}\StringTok{"Empirical logit"}\NormalTok{)  }
\end{Highlighting}
\end{Shaded}

\begin{figure}[H]

{\centering \includegraphics{03_model-fitting_files/figure-pdf/fig-elogit-elev-malkenya-1.pdf}

}

\caption{\label{fig-elogit-elev-malkenya}Plot of the empirical logit
against elevation measured in meters.}

\end{figure}

The resulting plot in Figure~\ref{fig-elogit-elev-malkenya} shows an
approximately linear relationship with decreasing values of the
empirical logit for increasing elevation. This is expected because the
cooler environment at higher altitudes is less favourable to the
development of the overall mosquito life cycle.

An alternative approach to generate a scatter plot for assessing the
association between elevation and the empirical logit would be to
aggregate the data at household level, rather than using classes of
elevation. However, this approach does not work as the one illustrated
above when only one individual is sampled for each location. In the case
of the \texttt{malkenya} data, the great majority of the locations only
include one individual making this second approach less useful than the
one illustrated.

Other more sophisticated approaches for the exploration of the
associations between covariates and binary outcomes are available. For
example, the use of the empirical logit could be avoided by using
non-parametric regression methods for Binomial outcomes (Bowman 1997),
also implemented in \texttt{sm} package in R. Our view is that a careful
exploratory analysis based on simpler methods, as those illustrated
above, can be equally effective to inform the module formulation.

\hypertarget{sec-overdispersion}{%
\subsection{Exploring overdispersion in count
data}\label{sec-overdispersion}}

One of the main advantages in the use of covariates is the ability to
attribute part of the variation of the outcome to a set of measured
variables and, hence, reduce the uncertainty of our inferences. However,
it almost always the case that the finite number of covariates at our
disposal is not enough to fully explain the variation in the outcome. In
other words, the existence of unmeasured covariates that are related to
the modelled outcome give rise to the so called residual variation. In a
standard linear regression model the extent to which we are able to
account for important covariates is directly linked to the size of the
variance of the residuals. In the case of count data, instead, this link
is less well defined and one of the main consequences of the omission of
covariates, which we address in this chapter, is \emph{overdispersion}.

Overdispersion occurs when the variability of the data is larger than
that implied by the generalized linear model (GLM) fitted to them. For
example, if we consider the Binomial distribution, the presence of
overdispersion implies that \(V(Y_i) > n_i \mu_{i}(1-\mu_i)\), where we
recall that \(n_i\) is the Binomial denominator and \(mu_i\) is the
probability of ``success'' for each of the \(n_i\) Bernoulli trials; for
a Poisson distribution with \(E(Y_i) = \mu_i\), instead, overdispersion
implies that \(V(Y_i) > \mu_{i}\).

Assessment of the overdispersion for count data can be carried out in
different ways depending on the goal of the statistical analysis. Since
the focus of this book is to illustrate how to formulate and apply
geostatistical models, the most natural approach to assess
overdispersion is through the use of generalized linear mixed models
(GLMMs). The class of GLMMs that we consider in this and the next
section are obtained by replacing the spatial Gaussian process
\(S(x_i)\) in introduced in Equation~\ref{eq-linear-predictor-ch1} with
a set of mutually independent random effects, which we denote as
\(Z_i\), and thus write
\begin{equation}\protect\hypertarget{eq-linear-predictor-glmms-ch2}{}{
g(\mu_i) = d(x_i)^\top \beta + Z_i.
}\label{eq-linear-predictor-glmms-ch2}\end{equation} The model above
accounts for the overdispersion in the data through \(Z_i\) whose
variance can be interpreted as an indicator of the amount of
overdispersion. To show this, we carry out a small simulation as
follows. For simplicity, we consider the Binomial mixed model with an
intercept only, hence
\begin{equation}\protect\hypertarget{eq-bin-glmm-example}{}{
\log\left\{\frac{\mu_i}{1-\mu_i}\right\} = \beta_0 + Z_i
}\label{eq-bin-glmm-example}\end{equation} and assume that the \(Z_i\)
follow a set of mutually independent Gaussian variables with mean 0 and
variance \(\tau^2\). In our simulation we vary \(\beta_0\) over the set
\(\{-3, -2, -1, 0, 1, 2, 3\}\) and set \(\tau^2=0.1\) and the binomial
denominators to \(n_i = 100\). For a given value of \(\beta_0\), we then
proceed through the following iterative steps.

\begin{itemize}
\item
  Simulate 10,000 values for \(Z_i\) from a Gaussian distribution with
  mean 0 and variance \(\tau^2\).
\item
  Compute the probabilities \(\mu_i\) based on
  Equation~\ref{eq-bin-glmm-example}.
\item
  Simulate 10,000 values from a Binomial model with probability of
  success \(\mu_i\) and denominator \(n_i\).
\item
  Compute the empirical variance of the counts \(y_i\) simulated in the
  previous step.
\item
  Change the value of \(\beta_0\) and repeat the previous steps, for all
  the values of \(\beta_0\).
\end{itemize}

The code below shows the implementation of the above steps in R.

\begin{Shaded}
\begin{Highlighting}[]
\CommentTok{\# Number of simulations}
\NormalTok{n\_sim }\OtherTok{\textless{}{-}} \DecValTok{10000}

\CommentTok{\# Variance of the Z\_i}
\NormalTok{tau2 }\OtherTok{\textless{}{-}} \FloatTok{0.1}

\CommentTok{\# Binomial denominator }
\NormalTok{bin\_denom }\OtherTok{\textless{}{-}} \DecValTok{100}

\CommentTok{\# Intercept values}
\NormalTok{beta0 }\OtherTok{\textless{}{-}} \FunctionTok{c}\NormalTok{(}\SpecialCharTok{{-}}\DecValTok{3}\NormalTok{, }\SpecialCharTok{{-}}\DecValTok{2}\NormalTok{, }\SpecialCharTok{{-}}\DecValTok{1}\NormalTok{, }\DecValTok{0}\NormalTok{, }\DecValTok{1}\NormalTok{, }\DecValTok{2}\NormalTok{, }\DecValTok{3}\NormalTok{)}

\CommentTok{\# Vector where we store the computed variance from}
\CommentTok{\# the simulated counts from the Binomial mixed model}
\NormalTok{var\_data }\OtherTok{\textless{}{-}} \FunctionTok{rep}\NormalTok{(}\ConstantTok{NA}\NormalTok{, }\FunctionTok{length}\NormalTok{(beta0))}


\ControlFlowTok{for}\NormalTok{(j }\ControlFlowTok{in} \DecValTok{1}\SpecialCharTok{:}\FunctionTok{length}\NormalTok{(beta0)) \{}
  \CommentTok{\# Simulation of the random effects Z\_i}
\NormalTok{  Z\_i\_sim }\OtherTok{\textless{}{-}} \FunctionTok{rnorm}\NormalTok{(n\_sim, }\AttributeTok{sd =} \FunctionTok{sqrt}\NormalTok{(tau2))}

  \CommentTok{\# Linear predictor of the Binomial mixed model}
\NormalTok{  lp }\OtherTok{\textless{}{-}}\NormalTok{ beta0[j]  }\SpecialCharTok{+}\NormalTok{ Z\_i\_sim}
  
  \CommentTok{\# Probabilities of the Binomial distribution conditional on Z\_i}
\NormalTok{  prob\_sim }\OtherTok{\textless{}{-}} \FunctionTok{exp}\NormalTok{(lp)}\SpecialCharTok{/}\NormalTok{(}\DecValTok{1}\SpecialCharTok{+}\FunctionTok{exp}\NormalTok{(lp))}
  
  \CommentTok{\# Simulation of the counts from the Binomial mixed model}
\NormalTok{  y\_i\_sim }\OtherTok{\textless{}{-}} \FunctionTok{rbinom}\NormalTok{(n\_sim, }\AttributeTok{size =}\NormalTok{ bin\_denom, }\AttributeTok{prob =}\NormalTok{ prob\_sim)}
  
  \CommentTok{\# Empirical variance from the simulated counts}
\NormalTok{  var\_data[j] }\OtherTok{\textless{}{-}} \FunctionTok{var}\NormalTok{(y\_i\_sim)}
\NormalTok{\}}

\CommentTok{\# Probabilities from the standard Binomial model (Z\_i = 0)}
\NormalTok{probs\_binomial }\OtherTok{\textless{}{-}} \FunctionTok{exp}\NormalTok{(beta0)}\SpecialCharTok{/}\NormalTok{(}\DecValTok{1}\SpecialCharTok{+}\FunctionTok{exp}\NormalTok{(beta0))}

\CommentTok{\# Variance from the standard Binomial model}
\NormalTok{var\_bimomial }\OtherTok{\textless{}{-}}\NormalTok{ bin\_denom}\SpecialCharTok{*}\NormalTok{probs\_binomial}\SpecialCharTok{*}\NormalTok{(}\DecValTok{1}\SpecialCharTok{{-}}\NormalTok{probs\_binomial)}
\end{Highlighting}
\end{Shaded}

\begin{Shaded}
\begin{Highlighting}[]
\FunctionTok{matplot}\NormalTok{(beta0, }\FunctionTok{cbind}\NormalTok{(var\_data, var\_bimomial), }\AttributeTok{type =} \StringTok{"b"}\NormalTok{, }\AttributeTok{pch =} \DecValTok{20}\NormalTok{,}
        \AttributeTok{lty =} \StringTok{"solid"}\NormalTok{, }\AttributeTok{ylab =} \StringTok{"Variance"}\NormalTok{, }\AttributeTok{xlab =} \FunctionTok{expression}\NormalTok{(beta[}\DecValTok{0}\NormalTok{]))}
\FunctionTok{legend}\NormalTok{(}\SpecialCharTok{{-}}\DecValTok{3}\NormalTok{, }\DecValTok{80}\NormalTok{, }\FunctionTok{c}\NormalTok{(}\StringTok{"Binomial mixed model"}\NormalTok{, }\StringTok{"Standard Binomial model"}\NormalTok{),}
       \AttributeTok{col=}\DecValTok{1}\SpecialCharTok{:}\DecValTok{2}\NormalTok{, }\AttributeTok{lty =} \StringTok{"solid"}\NormalTok{, }\AttributeTok{cex =} \FloatTok{0.75}\NormalTok{)}
\end{Highlighting}
\end{Shaded}

\begin{figure}[H]

{\centering \includegraphics{03_model-fitting_files/figure-pdf/fig-var-bin-glmm-1.pdf}

}

\caption{\label{fig-var-bin-glmm}Plot of the variances of the standard
Binomial model and the Binomial mixed model (see
Equation~\ref{eq-bin-glmm-example}) against \(\beta_0\)}

\end{figure}

Figure~\ref{fig-var-bin-glmm} shows the results of the simulation. In
this figure, the red line corresponds to the variance of a standard
Binomial model, obtained by setting \(Z_i=0\) and computed as
\(n_i \mu_i (1-\mu_i)\) with
\(\mu_i = \exp\{\beta_0\}/(1+\exp\{\beta_0\})\). As expected, this plot
shows that the variance of the simulated counts from the mixed model in
Equation~\ref{eq-bin-glmm-example} exhibit a larger variance than would
be expected under the standard Binomial model. It also indicates that
the chosen value for the variance of \(Z_i\) of \(\tau^2 = 0.1\)
corresponds to a significant amount of dispersion. One way to relate
\(\tau^2\) to the amount of overdispersion is by considering that,
following from the properties of a univariate Gaussian distribution,
\emph{a priori} the \(Z_i\) will take values between
\(-1.96 \sqrt{\tau^2}\) and \(+1.96 \sqrt{\tau^2}\) with approximately
95\(\%\) probability. That implies that \(\exp\{Z_i\}\), which expresses
the effect of the random effects on the odds ratios, will be with
95\(\%\) probability between \(\exp\{-1.96 \sqrt{\tau^2}\}\) and
\(\exp\{+1.96 \sqrt{\tau^2}\}\). By replacing \(\tau^2\) with the chosen
values for the simulation, those two becomes about 0.54 and 1.86,
meaning that with the \(Z_i\) with \(95\%\) probability will have a
multiplicative effect on the odds ratios between \(0.54\) and \(1.86\).

We encourage you to do Exercise 1 and Exercise 2 at the end of this
chapter, to further explore how generalized linear mixed models can be
used as a tool to account for overdispersion.

\hypertarget{maximum-likelihood-estimation-of-generalized-linear-mixed-models}{%
\subsubsection{Maximum likelihood estimation of generalized linear mixed
models}\label{maximum-likelihood-estimation-of-generalized-linear-mixed-models}}

We now illustrate how to fit a generalize linear mixed, using the
\texttt{anopheles} data-set as an example. We consider two models: an
intercept-only model and one that uses elevation as a covariate. Let
\(\mu(x_i)\) be the number of mosquitoes captured at a location \(x_i\);
then the linear predictor with elevation as a covariate takes the form
\begin{equation}\protect\hypertarget{eq-anopheles-glmm}{}{
\log\{\mu_i\} = \beta_{0} + \beta_{1} d(x_i) + Z_i
}\label{eq-anopheles-glmm}\end{equation} where \(d(x_i)\) indicates the
elevation in meters at location \(x_i\) and the \(Z_i\) are independent
and identically distributed Gaussian variables with mean 0 and variance
\(\tau^2\). The model with an intercept only is simply obtained by
setting \(\beta_1 = 0\).

We carry out the estimation in R using the \texttt{glmer} function from
the \texttt{lme4} package (see Bates et al. (2015) for a detailed
tutorial). The \texttt{glmer} function implements the maximum likelihood
estimation for generalized linear mixed models. The code below shows how
the \texttt{glmer} is used to carry out this step for the model in
Equation~\ref{eq-anopheles-glmm} and the one withuot covariates.

\begin{Shaded}
\begin{Highlighting}[]
\CommentTok{\# Create the ID of the location}
\NormalTok{anopheles}\SpecialCharTok{$}\NormalTok{ID\_loc }\OtherTok{\textless{}{-}} \DecValTok{1}\SpecialCharTok{:}\FunctionTok{nrow}\NormalTok{(anopheles)}

\CommentTok{\# Poisson mixed model with elevation}
\NormalTok{fit\_glmer\_elev }\OtherTok{\textless{}{-}} \FunctionTok{glmer}\NormalTok{(An.gambiae }\SpecialCharTok{\textasciitilde{}} \FunctionTok{scale}\NormalTok{(elevation) }\SpecialCharTok{+}\NormalTok{ (}\DecValTok{1}\SpecialCharTok{|}\NormalTok{ID\_loc), }\AttributeTok{family =}\NormalTok{ poisson, }
                        \AttributeTok{data =}\NormalTok{ anopheles, }\AttributeTok{nAGQ =} \DecValTok{25}\NormalTok{)}

\CommentTok{\# Poisson mixed model with intercept only}
\NormalTok{fit\_glmer\_int }\OtherTok{\textless{}{-}} \FunctionTok{glmer}\NormalTok{(An.gambiae }\SpecialCharTok{\textasciitilde{}}\NormalTok{ (}\DecValTok{1}\SpecialCharTok{|}\NormalTok{ID\_loc), }\AttributeTok{family =}\NormalTok{ poisson, }
                        \AttributeTok{data =}\NormalTok{ anopheles, }\AttributeTok{nAGQ =} \DecValTok{25}\NormalTok{)}
\end{Highlighting}
\end{Shaded}

To fit the model with \texttt{glmer}, we first must create a variable in
our data-set that allows us to identify the location associated with
each count. In this case, since every row corresponds to a different
location, we simply use the row number to identify the locations and
save this in the \texttt{ID\_loc} variable. The random effects \(Z_i\)
are then included in the model by adding
\texttt{(1\ \textbar{}\ ID\_loc)} in the formula of the \texttt{glmer}
function.

When introducing the variable elevation, we standardize the variable so
that its mean is 0 and its variance is 1. This is done to aid the
convergence of the algorithm used to fit the model and it is generally
considered good practice, especially when many variables with different
scales are used as covariates. However, we emphasize that standardizing
a variable does not affect the fit of the model to the data. This is
because the model with the standardized variable is a reparametrization
of the model with the unstandardized variable. In other words, a model
that uses standardized covariates only attaches a different
interpretation to its regression coefficients while maintaining the same
goodness of fit of the model with that uses the covariates on their
original scale.

The argument \texttt{nAGQ} is used to define the precision of the
approximation of the maximum likelihood estimation algorithm. By default
\texttt{nAGQ\ =\ 1}, which corresponds to the Laplace approximation.
Values for \texttt{nAGQ} larger than 1 are used to define the number of
points of the adaptive Gaussian-Hermite quadrature. The general
principle is that the larger \texttt{nAGQ} the better, but at the
expense of an increased computing time. Based on the guidelines and help
pages of the \texttt{lme4} package, it is stated that a reasonable value
for \texttt{nAGQ} is 25. For more technical details on this aspect, we
refer the reader to Bates et al. (2015).

We can now look at the summary of the fitted models to the mosquitoes
data-set.

\begin{Shaded}
\begin{Highlighting}[]
\DocumentationTok{\#\#\# Summary of the model with elevation}
\FunctionTok{summary}\NormalTok{(fit\_glmer\_elev)}
\DocumentationTok{\#\# Generalized linear mixed model fit by maximum likelihood (Adaptive}
\DocumentationTok{\#\#   Gauss{-}Hermite Quadrature, nAGQ = 25) [glmerMod]}
\DocumentationTok{\#\#  Family: poisson  ( log )}
\DocumentationTok{\#\# Formula: An.gambiae \textasciitilde{} scale(elevation) + (1 | ID\_loc)}
\DocumentationTok{\#\#    Data: anopheles}
\DocumentationTok{\#\# }
\DocumentationTok{\#\#      AIC      BIC   logLik deviance df.resid }
\DocumentationTok{\#\#    291.8    300.1   {-}142.9    285.8      113 }
\DocumentationTok{\#\# }
\DocumentationTok{\#\# Scaled residuals: }
\DocumentationTok{\#\#      Min       1Q   Median       3Q      Max }
\DocumentationTok{\#\# {-}0.89574 {-}0.42469 {-}0.09483  0.29445  0.53352 }
\DocumentationTok{\#\# }
\DocumentationTok{\#\# Random effects:}
\DocumentationTok{\#\#  Groups Name        Variance Std.Dev.}
\DocumentationTok{\#\#  ID\_loc (Intercept) 0.7146   0.8453  }
\DocumentationTok{\#\# Number of obs: 116, groups:  ID\_loc, 116}
\DocumentationTok{\#\# }
\DocumentationTok{\#\# Fixed effects:}
\DocumentationTok{\#\#                  Estimate Std. Error z value Pr(\textgreater{}|z|)    }
\DocumentationTok{\#\# (Intercept)       1.53042    0.09365  16.342   \textless{}2e{-}16 ***}
\DocumentationTok{\#\# scale(elevation) {-}0.19794    0.08950  {-}2.212    0.027 *  }
\DocumentationTok{\#\# {-}{-}{-}}
\DocumentationTok{\#\# Signif. codes:  0 \textquotesingle{}***\textquotesingle{} 0.001 \textquotesingle{}**\textquotesingle{} 0.01 \textquotesingle{}*\textquotesingle{} 0.05 \textquotesingle{}.\textquotesingle{} 0.1 \textquotesingle{} \textquotesingle{} 1}
\DocumentationTok{\#\# }
\DocumentationTok{\#\# Correlation of Fixed Effects:}
\DocumentationTok{\#\#             (Intr)}
\DocumentationTok{\#\# scale(lvtn) 0.036}

\DocumentationTok{\#\#\# Summary of the model with the intercept only}
\FunctionTok{summary}\NormalTok{(fit\_glmer\_int)}
\DocumentationTok{\#\# Generalized linear mixed model fit by maximum likelihood (Adaptive}
\DocumentationTok{\#\#   Gauss{-}Hermite Quadrature, nAGQ = 25) [glmerMod]}
\DocumentationTok{\#\#  Family: poisson  ( log )}
\DocumentationTok{\#\# Formula: An.gambiae \textasciitilde{} (1 | ID\_loc)}
\DocumentationTok{\#\#    Data: anopheles}
\DocumentationTok{\#\# }
\DocumentationTok{\#\#      AIC      BIC   logLik deviance df.resid }
\DocumentationTok{\#\#    294.6    300.1   {-}145.3    290.6      114 }
\DocumentationTok{\#\# }
\DocumentationTok{\#\# Scaled residuals: }
\DocumentationTok{\#\#      Min       1Q   Median       3Q      Max }
\DocumentationTok{\#\# {-}0.73816 {-}0.42718 {-}0.06941  0.26564  0.45022 }
\DocumentationTok{\#\# }
\DocumentationTok{\#\# Random effects:}
\DocumentationTok{\#\#  Groups Name        Variance Std.Dev.}
\DocumentationTok{\#\#  ID\_loc (Intercept) 0.761    0.8724  }
\DocumentationTok{\#\# Number of obs: 116, groups:  ID\_loc, 116}
\DocumentationTok{\#\# }
\DocumentationTok{\#\# Fixed effects:}
\DocumentationTok{\#\#             Estimate Std. Error z value Pr(\textgreater{}|z|)    }
\DocumentationTok{\#\# (Intercept)  1.52849    0.09584   15.95   \textless{}2e{-}16 ***}
\DocumentationTok{\#\# {-}{-}{-}}
\DocumentationTok{\#\# Signif. codes:  0 \textquotesingle{}***\textquotesingle{} 0.001 \textquotesingle{}**\textquotesingle{} 0.01 \textquotesingle{}*\textquotesingle{} 0.05 \textquotesingle{}.\textquotesingle{} 0.1 \textquotesingle{} \textquotesingle{} 1}
\end{Highlighting}
\end{Shaded}

From the summary of the model that uses elevation, we observe the that
the estimated regression coefficient \(\beta_{1}\) is statistically
significant different from 0. The interpretation of the estimated
regression coefficient is the following: for an increase of about 100
meters in elevation, all other things being equal, the average number of
mosquitoes decreases by about
\(100\% \times [1-\exp\{-0.19794\}] \approx 18\%\). Note that when using
a standardized variable, a unit increase for this corresponds to an
increase in the original unstandardized variable equal to its standard
deviation, which for the \texttt{elevation} variable is about 100
meters.

From the summaries of the two models, under \texttt{Random\ effects:},
we obtain the estimates associates with the random effects introduced in
the model. In this case, since we only have introduced \(Z_i\), this
part of summary provides the maximum likelihood estimate for \(\tau^2\),
the variance of \(Z_i\), which found on the line where \texttt{ID\_loc}
is printed. We then observe that the estimates for \(\tau^2\) for the
intercept-only model is \texttt{0.761}, whilst for the model with
elevation this is \texttt{0.7146}. Note that the figures reported under
\texttt{Std.Dev.} are simply the square root of the value reported under
\texttt{Variance}. As expected, the introduction of elevation
contributes to the explanation of the residual variation captured by
\(Z_i\), though by a very small amount. The estimated values of
\(\tau^2\) thus suggest that there is extra-Binomial variation in the
data that is not account for by elevation.

In the next section, we will illustrate how to assess the presence of
residual correlation for continuous measurements and overdispersed count
data.

\hypertarget{sec-expl-spatial}{%
\subsection{Exploring residual spatial
correlation}\label{sec-expl-spatial}}

In its most basic form, the concept of spatial correlation can be
succinctly encapsulated by Tobler (1970) first law of geography, which
posits that ``everything is interconnected, but objects in close
proximity exhibit stronger relationships than those situated farther
apart.'' After we have identified the key variables to introduce as
covariates in the model (Section~\ref{sec-expl-assoc}) and, in the case
of count data, assessed the presence of overdispersion
(Section~\ref{sec-overdispersion}), our final exploratory step consists
of assessing whether the residuals of the non spatial model show
evidence of spatial correlation. Hence, in geostatistical modelling, the
interest is not in the spatial correlation of the data, but rather on
understanding whether the variation in the outcome unexplained by the
covariates exhibits spatial correlation. We call this \emph{residual
spatial correlation}, to emphasize that spatial correlation is a concept
relative to the covariates that we have introduced in the model.

In the context of geostatistical analysis, the tool that is generally
used to assess the residual spatial correlation is the the so called
\emph{empirical variogram}. Before looking at the mathematical
definition of the empirical variogram, let us consider a generalized
linear mixed model as expressed in
Equation~\ref{eq-linear-predictor-glmms-ch2}. Our goal is then to
question the assumption of independently distributed random effects
\(Z_i\) by asking whether the \(Z_i\) show evidence of spatial
correlation. Let \(Z_i\) and \(Z_j\) be two random effects that are
associate with two different locations \(x_i\) and \(x_j\),
respectively, and let us take the squared difference between the two
\begin{equation}\protect\hypertarget{eq-squared-diff}{}{
V_{ij} = (Z_i - Z_j)^2.
}\label{eq-squared-diff}\end{equation} How does the spatial correlation
affect the value of \(V_{ij}\)? To answer this question, we can refer to
the aforementioned Tobler's law of geography. When \(x_i\) and \(x_j\)
will be closer to each other, then \(Z_i\) and \(Z_j\) will also tend to
be more similar to each other, thus making \(V_{ij}\) smaller, on
average. On the contrary, when \(x_i\) and \(x_j\) will be further
apart, then \(V_{ij}\) will become larger, on average. We can then
construct the empirical variogram by considering all possible pairs of
locations \(x_i\) and \(x_j\), for which we then compute \(V_{ij}\) and
plot this against the distance between \(x_i\) and \(x_j\), which we
denote as \(u_{ij}\). If there is spatial correlation in the random
effects \(Z_i\), then this will manifest as an average increase in the
\(V_{ij}\) as \(u_{ij}\) increases. However, there are still two issues
that we have to address before we can generate and plot the empirical
variogram.

The first issue is that we do not observe \(Z_i\) as, by definition,
this is a latent variable. Hence, we require an estimate for \(Z_i\)
which we can then feed into \(V_{ij}\). To emphasize this point, from
now on, we shall replace Equation~\ref{eq-squared-diff} with
\begin{equation}\protect\hypertarget{eq-hat-squared-diff}{}{
\hat{V}_{ij} = (\hat{Z}_{i} - \hat{Z}_j)^2.
}\label{eq-hat-squared-diff}\end{equation} Several options are available
for estimating \(Z_{i}\). Our choice is to use the model of the
predictive distribution of \(Z_i\), that is the distribution of
\(Z_{i}\) conditioned to the data \(y_i\). This estimator for \(Z_i\) is
also readily available from the output of the \texttt{lmer} and
\texttt{glmer} functions of the \texttt{lme4} package, as we will
illustrate later in our example in this section.

The second issue is that if simply plot \(\hat{V}_{ij}\) against the
distances \(u_{ij}\) (also knwon as \emph{cloud variogram}), due to the
high noiseness in the \(\hat{V}_{ij}\), it may be quite difficult to
assess the presence of an increasing trend in the \(\hat{V}_{ij}\) and
thus detect spatial correlation. Hence, it is general practice to group
the distances \(u_{ij}\) into classes, say \(\mathcal{U}\), and then
take average of all the \(\hat{V}_{ij}\) that fall within
\(\mathcal{U}\).

We can now write the formal definition of the empirical variogram as
\begin{equation}\protect\hypertarget{eq-empirical-variogram}{}{
\hat{V}(\mathcal{U}) = \frac{1}{2 |\mathcal{U}|} \sum_{(i, j): (u_i, u_j) \in \mathcal{U}} \hat{V}_{ij}
}\label{eq-empirical-variogram}\end{equation} where \(|\mathcal{U}|\)
denotes the number of pairs of locations that fall within the distance
class \(\mathcal{U}\). The rationale behind dividing by 2 in
\(1/2 |\mathcal{U}|\) from the above equation, will be elucidated in
Section~\ref{sec-linear-model}, and there is no need for us to delve
into this matter at this juncture. When creating the empirical variogram
plot, we select the midpoint values of the distance classes
\(\mathcal{U}\) to represent the x-axis values.

Before we can evaluate residual spatial correlation, there remains one
crucial concern: relying solely on a visual inspection of the empirical
variogram is susceptible to human subjectivity. Furthermore, it is worth
noting that even a seemingly upward trend observed in the empirical
variogram might be merely a product of random fluctuations, rather than
a reliable indication of actual residual spatial correlation. To address
these concerns and enhance the objectivity of the use of the empirical
variogram, one approach would involve comparing the observed empirical
variogram pattern with those generated in the absence of spatial
correlation. Following this principle, we then use a permutation test
that allows us to generate empirical variograms under the assumption of
absence of spatial correlation through the following iterative steps.

\begin{enumerate}
\def\labelenumi{\arabic{enumi}.}
\item
  Permute the order of the locations in the data-set while keeping
  everything else fixed.
\item
  Compute the empirical variogram \(\hat{V}(\mathcal{U})\) for the
  permuted data-set.
\item
  Repeat 1 and 2 a large number of times, say 10,000.
\item
  Use the resulting 10,000 empirical varigorams to compute 95\(\%\)
  confidence intervals, by taking the 0.025 and 0.975 quantiles of these
  for each distance class \(\hat{V}(\mathcal{U})\).
\item
  If the observed empirical variogram falls fully within the envelope
  generated in the previous point, we then conclude that the data do not
  exhibit residual spatial correlation. If, instead, the observed
  empirical variogram partly falls outside the envelope we conclude that
  the data do exhibit residual spatial correlation.
\end{enumerate}

We now show an application of all the concepts introduced in this
section to the Liberia data on river-blindness.

\hypertarget{sec-empirical-variog}{%
\subsubsection{Using the empirical variogram to assess spatial
correlation for the Liberia data}\label{sec-empirical-variog}}

We consider the Binomial mixed model that uses the log-transformed
elevation as a covariate to model river blindness prevalence, hence
\begin{equation}\protect\hypertarget{eq-liberia-bin-mixed-elev}{}{
\log\left\{\frac{p(x_i)}{1-p(x_i)}\right\} = \beta_{0} + \beta_{1}\log\{e(x_i)\} + Z_i
}\label{eq-liberia-bin-mixed-elev}\end{equation} where \(e(x_i)\) is the
elevation in meters at location \(x_i\) and the \(Z_i\) are i.i.d.
Gaussian variables with mean 0 and variance \(\tau^2\). We first fit the
model above using the \texttt{glmer} function.

\begin{Shaded}
\begin{Highlighting}[]
\CommentTok{\# Convert the data{-}set into an sf object}
\NormalTok{liberia }\OtherTok{\textless{}{-}} \FunctionTok{st\_as\_sf}\NormalTok{(liberia, }\AttributeTok{coords =} \FunctionTok{c}\NormalTok{(}\StringTok{"lat"}\NormalTok{, }\StringTok{"long"}\NormalTok{), }\AttributeTok{crs =} \DecValTok{4326}\NormalTok{)}


\CommentTok{\# Create the ID of the location}
\NormalTok{liberia}\SpecialCharTok{$}\NormalTok{ID\_loc }\OtherTok{\textless{}{-}} \DecValTok{1}\SpecialCharTok{:}\FunctionTok{nrow}\NormalTok{(liberia)}

\CommentTok{\# Binomial mixed model with log{-}elevation}
\NormalTok{fit\_glmer\_lib }\OtherTok{\textless{}{-}} \FunctionTok{glmer}\NormalTok{(}\FunctionTok{cbind}\NormalTok{(npos, ntest) }\SpecialCharTok{\textasciitilde{}} \FunctionTok{log}\NormalTok{(elevation) }\SpecialCharTok{+}\NormalTok{ (}\DecValTok{1}\SpecialCharTok{|}\NormalTok{ID\_loc), }\AttributeTok{family =}\NormalTok{ binomial,}
                        \AttributeTok{data =}\NormalTok{ liberia, }\AttributeTok{nAGQ =} \DecValTok{25}\NormalTok{)}

\FunctionTok{summary}\NormalTok{(fit\_glmer\_lib)}
\end{Highlighting}
\end{Shaded}

\begin{verbatim}
Generalized linear mixed model fit by maximum likelihood (Adaptive
  Gauss-Hermite Quadrature, nAGQ = 25) [glmerMod]
 Family: binomial  ( logit )
Formula: cbind(npos, ntest) ~ log(elevation) + (1 | ID_loc)
   Data: liberia

     AIC      BIC   logLik deviance df.resid 
   127.9    135.4    -61.0    121.9       87 

Scaled residuals: 
     Min       1Q   Median       3Q      Max 
-2.46033 -0.63341 -0.07633  0.61995  3.12732 

Random effects:
 Groups Name        Variance Std.Dev.
 ID_loc (Intercept) 0.003097 0.05565 
Number of obs: 90, groups:  ID_loc, 90

Fixed effects:
               Estimate Std. Error z value Pr(>|z|)    
(Intercept)    -2.96292    0.21184 -13.987  < 2e-16 ***
log(elevation)  0.26143    0.04071   6.422 1.35e-10 ***
---
Signif. codes:  0 '***' 0.001 '**' 0.01 '*' 0.05 '.' 0.1 ' ' 1

Correlation of Fixed Effects:
            (Intr)
log(elevtn) -0.981
\end{verbatim}

From the output, we observe that the estimate for \(\tau^2\) is about
0.003, indicating a moderate level of overdispersion in the data.

\begin{Shaded}
\begin{Highlighting}[]
\NormalTok{liberia}\SpecialCharTok{$}\NormalTok{Z\_hat }\OtherTok{\textless{}{-}} \FunctionTok{ranef}\NormalTok{(fit\_glmer\_lib)}\SpecialCharTok{$}\NormalTok{ID\_loc[,}\DecValTok{1}\NormalTok{]}
\end{Highlighting}
\end{Shaded}

Throuhg the function \texttt{ranef} when extract the estimates of the
random effects \(Z_i\) and save these in the data set. We then use the
function \texttt{s\_variogram} from the \texttt{RiskMap} package to
compute the empirical varigoram for the estimated \(\hat{Z}_i\).

\begin{Shaded}
\begin{Highlighting}[]
\NormalTok{liberia\_variog }\OtherTok{\textless{}{-}} \FunctionTok{s\_variogram}\NormalTok{(}\AttributeTok{data =}\NormalTok{ liberia,}
                              \AttributeTok{variable =} \StringTok{"Z\_hat"}\NormalTok{,}
                              \AttributeTok{bins =} \FunctionTok{c}\NormalTok{(}\DecValTok{15}\NormalTok{, }\DecValTok{30}\NormalTok{, }\DecValTok{40}\NormalTok{, }\DecValTok{80}\NormalTok{, }\DecValTok{120}\NormalTok{,}
                                       \DecValTok{160}\NormalTok{, }\DecValTok{200}\NormalTok{, }\DecValTok{250}\NormalTok{, }\DecValTok{300}\NormalTok{, }\DecValTok{350}\NormalTok{),}
                              \AttributeTok{scale\_to\_km =} \ConstantTok{TRUE}\NormalTok{,}
                              \AttributeTok{n\_permutation =} \DecValTok{10000}\NormalTok{)}
\end{Highlighting}
\end{Shaded}

Through the argument \texttt{bins} we can specify the the classes of
distance, previously denoted by \(\mathcal{U}\); check the help page of
\texttt{s\_variogram} to see how this is defined by default. The value
passed to \texttt{bins} in the code above correspond to define the
following classes of distance \(\mathcal{U}\): \([15, 30]\),
\((30, 40]\) and so forth, with the last class being \([350, +\infty)\),
i.e.~all pairs of locations whose distances are above 350km. The
argument \texttt{n\_permutation} allows the user to specify the number
of permutations that are performed the generate the envelope for absence
of spatial correlation previously described.

\begin{Shaded}
\begin{Highlighting}[]
\FunctionTok{dist\_summaries}\NormalTok{(}\AttributeTok{data =}\NormalTok{ liberia,}
               \AttributeTok{scale\_to\_km =} \ConstantTok{TRUE}\NormalTok{)}
\DocumentationTok{\#\# $min}
\DocumentationTok{\#\# [1] 3.34536}
\DocumentationTok{\#\# }
\DocumentationTok{\#\# $max}
\DocumentationTok{\#\# [1] 533.0733}
\DocumentationTok{\#\# }
\DocumentationTok{\#\# $mean}
\DocumentationTok{\#\# [1] 206.7424}
\DocumentationTok{\#\# }
\DocumentationTok{\#\# $median}
\DocumentationTok{\#\# [1] 192.6496}
\end{Highlighting}
\end{Shaded}

The \texttt{dist\_summaries} function within the \texttt{RiskMap}
package can be used for gauging the extent of the area covered by your
dataset, aiding in the selection of appropriate values to be passed to
the \texttt{bins} argument. In the provided output above, we can observe
that for the Liberia dataset, the minimum and maximum distances span
approximately 3km and 533km, respectively. While there is not a
one-size-fits-all recommendation for setting \texttt{bins}, two
fundamental principles should inform your decision-making. Firstly, it
is advisable to avoid choosing overly large distance intervals, as the
uncertainty associated with the empirical variogram tends to increase
with distance due to fewer available pairs of observations for
estimation. Secondly, especially when spatial correlation is not strong,
it is crucial to carefully explore the behavior of the variogram at
smaller distances. Consequently, it is generally advisable to experiment
with different \texttt{bins} configurations and observe how they impact
the pattern of the empirical variogram.

\begin{Shaded}
\begin{Highlighting}[]
\FunctionTok{plot\_s\_variogram}\NormalTok{(liberia\_variog,}
                 \AttributeTok{plot\_envelope =} \ConstantTok{TRUE}\NormalTok{)}
\end{Highlighting}
\end{Shaded}

\begin{figure}[H]

{\centering \includegraphics{03_model-fitting_files/figure-pdf/fig-liberia-variog-1.pdf}

}

\caption{\label{fig-liberia-variog}Plot of the empirical variogram
(solid line) computed using the estimated random effects from the model
in Equation~\ref{eq-liberia-bin-mixed-elev}. The blue shaded area is the
95\% confidence level envelope generated using the permutation procedure
described in Section~\ref{sec-expl-spatial}.}

\end{figure}

Finally, the \texttt{plot\_s\_variogram} function enables us to
visualize the empirical variogram and, through the
\texttt{plot\_envelope} argument, include the envelope generated by the
permutation procedure. As illustrated in
Figure~\ref{fig-liberia-variog}, we observe that the empirical variogram
falls outside the envelope at relatively short distances, typically
below 30km. However, for distances exceeding 30km, the behavior of the
empirical variogram does not significantly differ from variograms
generated under the assumption of spatial independence. In summary, we
interpret the evidence presented in Figure~\ref{fig-liberia-variog} as
indicative of residual spatial correlation within the data.
Nevertheless, it is essential to exercise caution when attempting to
ascertain the scale of the spatial correlation using the empirical
variogram. As we will emphasize throughout this book, the empirical
variogram's sensitivity to the choice of bins values renders it an
unreliable tool for drawing statistical inferences. In other words, we
advocate employing the empirical variogram primarily to assess the
presence of residual correlation.

\hypertarget{sec-linear-model}{%
\section{Linear Gaussian model}\label{sec-linear-model}}

\hypertarget{generalized-linear-geostatistical-models}{%
\section{Generalized linear geostatistical
models}\label{generalized-linear-geostatistical-models}}

\hypertarget{theory}{%
\section{Theory}\label{theory}}

\hypertarget{the-likelihood-function-of-a-genearlized-linear-mixed-model}{%
\subsection{The likelihood function of a genearlized linear mixed
model}\label{the-likelihood-function-of-a-genearlized-linear-mixed-model}}

\hypertarget{exercises}{%
\section{Exercises}\label{exercises}}

\begin{enumerate}
\def\labelenumi{\arabic{enumi}.}
\setcounter{enumi}{1}
\item
  Consider the Binomial mixed model with linear predictor as defined in
  Equation~\ref{eq-linear-predictor-glmms-ch2}. By editing the code for
  the simulation shown in Section~\ref{sec-overdispersion}, generate a
  graph as in Figure~\ref{fig-var-bin-glmm} under the two following
  scenarios: \(i\)) \(\tau^2 = 0.2\) and \(n_i=100\); \(ii\))
  \(\tau^2 = 0.1\) and \(n_i = 1\). How does the variance of \(Y_i\)
  change under \(i\)) and \(ii\)) in comparison to
  Figure~\ref{fig-var-bin-glmm}? How do you explain the differences?
\item
  Similarly to the previous exercise, consider a Poisson mixed model
  with linear predictor \[
  \log\left\{\mu_i\right\} = \beta_0 + Z_i,
  \] where \(Z_i\) are a set of mutually independent Gaussian variables
  with mean 0 and variance \(\tau^2\). Using the code shown in
  Section~\ref{sec-overdispersion}, carry out a simulation study to
  compute the variance of \(Y_i\) and generate a graph similar to
  Figure~\ref{fig-var-bin-glmm} to compare the variance of the Poisson
  mixed model with that of a standard Poisson model. Generate the graph
  for different values of \(\tau^2\) and summarize your findings. NOTE:
  In this simulation the offset \(n_i\) can be set to 1.
\item
  Create an R function that computes the \emph{cloud variogram}. As
  explained in Section~\ref{sec-expl-spatial}, the cloud variogram is
  obtained by plotting \(\hat{V}_ij\) (see
  Equation~\ref{eq-hat-squared-diff}) against the distances \(u_ij\).
  The function should take as input a data-set with three columns: the
  variable for which the cloud variogram is to be computed; and two
  columns corresponding to the location of the data. Then, use this
  function to create the cloud variogram for the model for
  river-blindness in Equation~\ref{eq-liberia-bin-mixed-elev}. How does
  this compare to empirical variogram that takes the averages within
  predefined distance classes, as shown in
  Figure~\ref{fig-liberia-variog}?
\item
  Fit a Binomial mixed model to the Liberia data-set on river-blindness
  without any covariates, i.e. \[
  \log\left\{\frac{p(x_i)}{1-p(x_i)}\right\} = \beta_0 + Z_i.
  \] Making use of the R code presented in
  Section~\ref{sec-empirical-variog}, use the function
  \texttt{s\_variogram} to generate the empirical variogram for this
  model and compare this to the empirical variogram of
  Figure~\ref{fig-liberia-variog}. What differences do you observe?
\end{enumerate}

\bookmarksetup{startatroot}

\hypertarget{sec-validation}{%
\chapter{Model validation}\label{sec-validation}}

This is a book created from markdown and executable code.

See (\textbf{knuth84?}) for additional discussion of literate
programming.

\begin{Shaded}
\begin{Highlighting}[]
\DecValTok{1} \SpecialCharTok{+} \DecValTok{1}
\end{Highlighting}
\end{Shaded}

\begin{verbatim}
[1] 2
\end{verbatim}

\hypertarget{how-to-simulate-geostatistical-data-from-a-fitted-model}{%
\section{How to simulate geostatistical data from a fitted
model}\label{how-to-simulate-geostatistical-data-from-a-fitted-model}}

\hypertarget{validating-the-calibration-of-the-model}{%
\section{Validating the calibration of the
model}\label{validating-the-calibration-of-the-model}}

\hypertarget{validating-the-spatial-correlation-of-the-model}{%
\section{Validating the spatial correlation of the
model}\label{validating-the-spatial-correlation-of-the-model}}

\bookmarksetup{startatroot}

\hypertarget{sec-geo-prediction}{%
\chapter{Geostatistical prediction}\label{sec-geo-prediction}}

This is a book created from markdown and executable code.

See (\textbf{knuth84?}) for additional discussion of literate
programming.

\begin{Shaded}
\begin{Highlighting}[]
\DecValTok{1} \SpecialCharTok{+} \DecValTok{1}
\end{Highlighting}
\end{Shaded}

\begin{verbatim}
[1] 2
\end{verbatim}

\hypertarget{pixel-level-predictive-targets}{%
\section{Pixel-level predictive
targets}\label{pixel-level-predictive-targets}}

\hypertarget{area-level-predictive-targets}{%
\section{Area-level predictive
targets}\label{area-level-predictive-targets}}

\hypertarget{comparing-the-predictive-performance-of-geostatistical-models}{%
\section{Comparing the predictive performance of geostatistical
models}\label{comparing-the-predictive-performance-of-geostatistical-models}}

\bookmarksetup{startatroot}

\hypertarget{sec-case-studies}{%
\chapter{Case studies}\label{sec-case-studies}}

This is a book created from markdown and executable code.

See (\textbf{knuth84?}) for additional discussion of literate
programming.

\begin{Shaded}
\begin{Highlighting}[]
\DecValTok{1} \SpecialCharTok{+} \DecValTok{1}
\end{Highlighting}
\end{Shaded}

\begin{verbatim}
[1] 2
\end{verbatim}

\hypertarget{mapping-stunting-risk-in-ghan}{%
\section{Mapping stunting risk in
Ghan}\label{mapping-stunting-risk-in-ghan}}

\hypertarget{mapping-river-blindness-in-malawi}{%
\section{Mapping river blindness in
Malawi}\label{mapping-river-blindness-in-malawi}}

\hypertarget{mapping-mosquitoes-abundance-in-cameroon}{%
\section{Mapping mosquitoes abundance in
Cameroon}\label{mapping-mosquitoes-abundance-in-cameroon}}

\bookmarksetup{startatroot}

\hypertarget{references}{%
\chapter*{References}\label{references}}
\addcontentsline{toc}{chapter}{References}

\markboth{References}{References}

\hypertarget{refs}{}
\begin{CSLReferences}{1}{0}
\leavevmode\vadjust pre{\hypertarget{ref-bates2015}{}}%
Bates, Douglas, Martin Mächler, Ben Bolker, and Steve Walker. 2015.
{``Fitting Linear Mixed-Effects Models Using {lme4}.''} \emph{Journal of
Statistical Software} 67 (1): 1--48.
\url{https://doi.org/10.18637/jss.v067.i01}.

\leavevmode\vadjust pre{\hypertarget{ref-bowman1997}{}}%
Bowman, A. W. 1997. \emph{Applied Smoothing Techniques for Data Analysis
: The Kernel Approach with s-Plus Illustrations}. Oxford Statistical
Science Series ; 18. Oxford : New York: Clarendon Press ; Oxford
University Press.

\leavevmode\vadjust pre{\hypertarget{ref-breslow1993}{}}%
Breslow, N. E., and D. G. Clayton. 1993. {``Approximate Inference in
Generalized Linear Mixed Models.''} \emph{Journal of the American
Statistical Association} 88: 9--25.

\leavevmode\vadjust pre{\hypertarget{ref-chilesdelfiner2016}{}}%
Chilès, J-P, and P. Delfiner. 2016. \emph{Geostatistics (Second
Edition)}. Hoboken: Wiley.

\leavevmode\vadjust pre{\hypertarget{ref-cressie1991}{}}%
Cressie, N. A. C. 1991. \emph{Statistics for Spatial Data}. New York:
Wiley.

\leavevmode\vadjust pre{\hypertarget{ref-diggle1998}{}}%
Diggle, P. J., J. A. Tawn, and R. A. Moyeed. 1998. {``Model-Based
Geostatistics.''} \emph{Journal of the Royal Statistical Society: Series
C (Applied Statistics)} 47 (3): 299--350.
\url{https://doi.org/10.1111/1467-9876.00113}.

\leavevmode\vadjust pre{\hypertarget{ref-diggleBook2019}{}}%
Diggle, Peter J. 2019. \emph{Model-Based Geostatistics for Global Public
Health : Methods and Applications.} Chapman and Hall/CRC
Interdisciplinary Statistics Ser. Milton: Chapman; Hall/CRC.

\leavevmode\vadjust pre{\hypertarget{ref-dobson2008}{}}%
Dobson, A. J., and A. Barnett. 2008. \emph{An Introduction to
Generalized Linear Models}. Third. Chapman; Hall/CRC.

\leavevmode\vadjust pre{\hypertarget{ref-fernandez2000}{}}%
Fernández, J. A, A Rey, and A Carballeira. 2000. {``An Extended Study of
Heavy Metal Deposition in Galicia (NW Spain) Based on Moss Analysis.''}
\emph{Science of The Total Environment} 254 (1): 31--44.
\url{https://doi.org/10.1016/S0048-9697(00)00431-9}.

\leavevmode\vadjust pre{\hypertarget{ref-hastie2001}{}}%
Hastie, Trevor, Robert Tibshirani, and Jerome Friedman. 2001. \emph{The
Elements of Statistical Learning}. Springer Series in Statistics. New
York, NY, USA: Springer New York Inc.

\leavevmode\vadjust pre{\hypertarget{ref-gates2020}{}}%
Katz, Elizabeth, and Bill \& Melinda Gates Foundation. 2020. {``Gender
and Malaria Evidence Reivew.''} Bill \& Melinda Gates Foundation.
\url{https://www.gatesgenderequalitytoolbox.org/wp-content/uploads/BMGF_Malaria-Review_FC.pdf}.

\leavevmode\vadjust pre{\hypertarget{ref-krige1951}{}}%
Krige, D. G. 1951. {``A Statistical Approach to Some Basic Mine
Valuation Problems on the Witwatersrand.''} \emph{Journal of the
Chemical, Metallurgical and Mining Society of South Africa} 52: 119--39.

\leavevmode\vadjust pre{\hypertarget{ref-matheron1963}{}}%
Matheron, G. 1963. {``Principles of Geostatistics.''} \emph{Economic
Geology} 58: 1246--66.

\leavevmode\vadjust pre{\hypertarget{ref-nelder1972}{}}%
Nelder, J. A., and R. W. M. Wedderburn. 1972. {``Generalized Linear
Models.''} \emph{Journal of the Royal Statistical Society A} 135:
370--84.

\leavevmode\vadjust pre{\hypertarget{ref-pawitan2001}{}}%
Pawitan, Yudi. 2001. \emph{In All Likelihood : Statistical Modelling and
Inference Using Likelihood}. Oxford ; New York: Clarendon Press : Oxford
University Press.

\leavevmode\vadjust pre{\hypertarget{ref-ripley1981}{}}%
Ripley, B. D. 1981. \emph{Spatial Statistics}. New York: Wiley.

\leavevmode\vadjust pre{\hypertarget{ref-ross2013}{}}%
Ross, Sheldon. 2013. \emph{First Course in Probability, a.} 9th ed.
Harlow: Pearson Education UK.

\leavevmode\vadjust pre{\hypertarget{ref-smith2007}{}}%
Smith, David L, Carlos A Guerra, Robert W Snow, and Simon I Hay. 2007.
{``Standardizing Estimates of the Plasmodium Falciparum Parasite
Rate.''} \emph{Malaria Journal} 6 (1): 131--31.

\leavevmode\vadjust pre{\hypertarget{ref-stevenson2013}{}}%
Stevenson, Gillian H. AND Gitonga, Jennifer C. AND Stresman. 2013.
{``Reliability of School Surveys in Estimating Geographic Variation in
Malaria Transmission in the Western Kenyan Highlands.''} \emph{PLOS ONE}
8 (10). \url{https://doi.org/10.1371/journal.pone.0077641}.

\leavevmode\vadjust pre{\hypertarget{ref-fossog2015}{}}%
Tene Fossog, Billy, Diego Ayala, Pelayo Acevedo, Pierre Kengne, Ignacio
Ngomo Abeso Mebuy, Boris Makanga, Julie Magnus, et al. 2015. {``Habitat
Segregation and Ecological Character Displacement in Cryptic African
Malaria Mosquitoes.''} \emph{Evolutionary Applications} 8 (4): 326--45.
\url{https://doi.org/10.1111/eva.12242}.

\leavevmode\vadjust pre{\hypertarget{ref-tobler1970}{}}%
Tobler, W. R. 1970. {``A Computer Movie Simulating Urban Growth in the
Detroit Region.''} \emph{Economic Geography} 46: 234--40.

\leavevmode\vadjust pre{\hypertarget{ref-watson1971}{}}%
Watson, G. S. 1971. {``Trend -Surface Analysis.''} \emph{Mathematical
Geology} 3: 215--26.

\leavevmode\vadjust pre{\hypertarget{ref-watson1972}{}}%
---------. 1972. {``Trend Surface Analysis and Spatial Correlation.''}
\emph{Geological Society of America Special Paper} 146: 39--46.

\leavevmode\vadjust pre{\hypertarget{ref-weisberg2014}{}}%
Weisberg, Sanford. 2014. \emph{Applied Linear Regression}. Fourth.
Hoboken {NJ}: Wiley. \url{http://z.umn.edu/alr4ed}.

\leavevmode\vadjust pre{\hypertarget{ref-zoure2014}{}}%
Zouré, Honorat GM, Mounkaila Noma, Afework H Tekle, Uche V Amazigo,
Peter J Diggle, Emanuele Giorgi, and Jan HF Remme. 2014. {``Geographic
Distribution of Onchocerciasis in the 20 Participating Countries of the
African Programme for Onchocerciasis Control: (2) Pre-Control Endemicity
Levels and Estimated Number Infected.''} \emph{Parasites \& Vectors} 7
(1): 326--26.

\end{CSLReferences}



\end{document}
